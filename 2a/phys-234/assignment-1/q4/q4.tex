\documentclass[11pt]{article}

\usepackage{amsmath}
\usepackage{braket}
\usepackage{booktabs}
\usepackage[outline]{contour}
\usepackage{enumitem}
\usepackage[T1]{fontenc}
\usepackage[margin=1in]{geometry}
\usepackage{graphicx}
\usepackage[utf8]{inputenc}
\usepackage{libertine}
\usepackage{mathtools}
\usepackage[libertine]{newtxmath}
\usepackage{parskip}
\usepackage[detect-weight=true]{siunitx}

\title{PHYS 234 Assignment 1}
\author{Brandon Tsang}
\date{May 18, 2020}

\contourlength{0.1em}

\DeclareMathOperator\conj{conj}
\newcommand\abs[1]{\lvert#1\rvert}
\newcommand\Abs[1]{\left\lvert#1\right\rvert}
\newcommand\uline[1]{\underline{\smash{#1}}\llap{\contour{white}{#1}}}

\newenvironment{amatrix}[1]{%
    \left[\begin{array}{@{}*{#1}{c}|c@{}}
}{%
    \end{array}\right]
}

\begin{document}
    \maketitle
    \begin{enumerate}[label=\textbf{\arabic*.}, start=4]
        \item{
            \textbf{\boldmath \uline{Matrix Operations} \\ Given the following matrices: \[A=\begin{bmatrix}-1 & 0 & i \\ 3 & 0 & 2 \\ i & -2i & 2\end{bmatrix}\qquad B=\begin{bmatrix}2 & 0 & i \\ 0 & i & 0 \\ 1 & 2 & 2\end{bmatrix}\] compute the following:}
                \begin{enumerate}[label=\textbf{(\alph*)}]
                    \item{
                        \textbf{\boldmath \(A+B\)}
                        \begin{align*}
                            A+B&=\begin{bmatrix}-1 & 0 & i \\ 3 & 0 & 2 \\ i & -2i & 2\end{bmatrix}+\begin{bmatrix}2 & 0 & i \\ 0 & i & 0 \\ 1 & 2 & 2\end{bmatrix} \\
                            &=\begin{bmatrix}1 & 0 & 2i \\ 3 & i & 2 \\ 1+i & 2-2i & 4\end{bmatrix}
                        \end{align*}
                    }
                    \item{
                        \textbf{\boldmath \(AB\)}
                        \begin{align*}
                            AB&=\begin{bmatrix}-1 & 0 & i \\ 3 & 0 & 2 \\ i & -2i & 2\end{bmatrix}\begin{bmatrix}2 & 0 & i \\ 0 & i & 0 \\ 1 & 2 & 2\end{bmatrix} \\
                            &=\begin{bmatrix}-2+i & 2i & i \\ 8 & 4 & 4+3i \\ 2+2i & 6 & 3\end{bmatrix}
                        \end{align*}
                    }
                    \item{
                        \textbf{\boldmath \([A,B]\) (commutator of \(A\) and \(B\))}
                        \par
                        The commutator of two matrices is defined as
                        \[[A,B]=AB-BA.\]
                        We already know \(AB\), so we must find \(BA\).
                        \begin{align*}
                            BA&=\begin{bmatrix}2 & 0 & i \\ 0 & i & 0 \\ 1 & 2 & 2\end{bmatrix}\begin{bmatrix}-1 & 0 & i \\ 3 & 0 & 2 \\ i & -2i & 2\end{bmatrix} \\
                            &=\begin{bmatrix}-3 & 2 & 4i \\ 3i & 0 & 2i \\ 5+2i & -4i & 8+i\end{bmatrix}
                        \end{align*}
                        Then,
                        \begin{align*}
                            [A,B]=AB-BA&=\begin{bmatrix}-2+i & 2i & i \\ 8 & 4 & 4+3i \\ 2+2i & 6 & 3\end{bmatrix}-\begin{bmatrix}-3 & 2 & 4i \\ 3i & 0 & 2i \\ 5+2i & -4i & 8+i\end{bmatrix} \\
                            &=\begin{bmatrix}-5+i & -2+2i & -3i \\ 8-3i & 4 & 4+i \\ -3 & 6+4i & -5-i\end{bmatrix}
                        \end{align*}
                    }
                    \item{
                        \textbf{\boldmath \(A^\mathrm{T}\) (transpose)}
                        \begin{equation*}
                            A^\mathrm{T}=\begin{bmatrix}-1 & 3 & i \\ 0 & 0 & -2i \\ i & 2 & 2\end{bmatrix}
                        \end{equation*}
                    }
                    \item{
                        \textbf{\boldmath \(A^\dagger\) (complex transpose)}
                        \begin{equation*}
                            A^\mathrm{T}=\begin{bmatrix}-1 & 3 & -i \\ 0 & 0 & 2i \\ -i & 2 & 2\end{bmatrix}
                        \end{equation*}
                    }
                    \item{
                        \textbf{\boldmath Verify by direct calculation that \((AB)^\mathrm{T}=B^\mathrm{T}A^\mathrm{T}\).}
                        \par
                        First, we calculate \((AB)^\mathrm{T}\):
                        \begin{equation*}
                            (AB)^\mathrm{T}=\begin{bmatrix}-2+i & 8 & 2+2i \\ 2i & 4 & 6 \\ i & 4+3i & 3\end{bmatrix}
                        \end{equation*}
                        Next, we calculate \(B^\mathrm{T}A^\mathrm{T}\):
                        \begin{align*}
                                B^\mathrm{T}A^\mathrm{T}&=\begin{bmatrix}2 & 0 & 1 \\ 0 & i & 2 \\ i & 0 & 2\end{bmatrix}\begin{bmatrix}-1 & 3 & i \\ 0 & 0 & -2i \\ i & 2 & 2\end{bmatrix} \\
                                &=\begin{bmatrix}-2+i & 8 & 2+2i \\ 2i & 4 & 6 \\ i & 4+3i & 3\end{bmatrix}
                        \end{align*}
                        We can see that they are the same, so it is true that \((AB)^\mathrm{T}=B^\mathrm{T}A^\mathrm{T}\).
                    }
                \end{enumerate}
        }
    \end{enumerate}
\end{document}
