\documentclass[11pt]{article}

\usepackage{amsmath}
\usepackage{braket}
\usepackage{booktabs}
\usepackage[outline]{contour}
\usepackage{enumitem}
\usepackage[T1]{fontenc}
\usepackage[margin=1in]{geometry}
\usepackage{graphicx}
\usepackage[utf8]{inputenc}
\usepackage{libertine}
\usepackage{mathtools}
\usepackage[libertine]{newtxmath}
\usepackage{parskip}
\usepackage[detect-weight=true]{siunitx}

\title{PHYS 234 Assignment 1}
\author{Brandon Tsang}
\date{May 18, 2020}

\contourlength{0.1em}

\DeclareMathOperator\conj{conj}
\newcommand\abs[1]{\lvert#1\rvert}
\newcommand\Abs[1]{\left\lvert#1\right\rvert}
\newcommand\uline[1]{\underline{\smash{#1}}\llap{\contour{white}{#1}}}

\newenvironment{amatrix}[1]{%
    \left[\begin{array}{@{}*{#1}{c}|c@{}}
}{%
    \end{array}\right]
}

\begin{document}
    \maketitle
    \begin{enumerate}[label=\textbf{\arabic*.}, start=2]
        \item{
            \textbf{\boldmath \uline{Calculations using quantum states} \begin{align*}\ket{\psi_1}&=3\ket{+}-i\ket{-} \\ \ket{\psi_2}&=e^{i\pi/3}\ket{+}+\ket{-} \\ \ket{\psi_3}&=7i\ket{+}-2\ket{-}\end{align*}}
            \begin{enumerate}[label=\textbf{(\alph*)}]
                \item{
                    \textbf{\boldmath For each of the states \(\ket{\psi_j}\) above (\(j=1,2,3\)), find the corresponding normalized state \(\ket{\psi_j}_\mathrm{N}\).}
                    \par
                    For \(\ket{\psi_1}\):
                    \begin{align*}
                        \braket{C\psi_1|C\psi_1}&=1 \\
                        1&=C^*(3\bra{+}+i\bra{-})\cdot C(3\ket{+}-i\ket{-}) \\
                        &=CC^*(9\braket{+|+}-3i\braket{+|-}+3i\braket{-|+}-i^2\braket{-|-}) \\
                        &=CC^*(9+1) \\
                        \abs{C}^2&=\frac{1}{10} \\
                        C&=\frac{1}{\sqrt{10}}
                    \end{align*}
                    Therefore, \(\ket{\psi_1}_\mathrm{N}=\frac{1}{\sqrt{10}}\ket{\psi_1}=\frac{3}{\sqrt{10}}\ket{+}-\frac{i}{\sqrt{10}}\ket{-}\).
                    \par
                    For \(\ket{\psi_2}\):
                    \begin{align*}
                        \braket{C\psi_2|C\psi_2}&=1 \\
                        1&=C^*\left(e^{-i\pi/3}\bra{+}+\bra{-}\right)\cdot C\left(e^{i\pi/3}\ket{+}+\ket{-}\right) \\
                        &=CC^*\left(\braket{+|+}+e^{-i\pi/3}\braket{+|-}+e^{i\pi/3}\braket{-|+}+\braket{-|-}\right) \\
                        &=CC^*(1+1) \\
                        \abs{C}^2&=\frac 1 2 \\
                        C&=\frac{1}{\sqrt 2}
                    \end{align*}
                    Therefore, \(\ket{\psi_2}_\mathrm{N}=\frac{1}{\sqrt 2}\ket{\psi_2}=\frac{1}{\sqrt 2}e^{i\pi/3}\ket{+}+\frac{1}{\sqrt 2}\ket{-}\).
                    \par
                    For \(\ket{\psi_3}\):
                    \begin{align*}
                        \braket{C\psi_3|C\psi_3}&=1 \\
                        1&=C^*(-7i\bra{+}-2\bra{-})\cdot C(7i\ket{+}-2\ket{-}) \\
                        &=CC^*(-49i^2\braket{+|+}+14i\braket{+|-}-14i\braket{-|+}+4\braket{-|-}) \\
                        &=CC^*(49+4) \\
                        \abs{C}^2&=\frac{1}{53} \\
                        C&=\frac{1}{\sqrt{53}}
                    \end{align*}
                    Therefore, \(\ket{\psi_3}_\mathrm{N}=\frac{1}{\sqrt{53}}\ket{\psi_3}=\frac{7i}{\sqrt{53}}\ket{+}-\frac{2}{\sqrt{53}}\ket{-}\).
                }
                \item{
                    \textbf{\boldmath Using the bra-ket notation, calculate all 9 inner products \(\prescript{}{\mathrm{N}}{\braket{\psi_i|\psi_j}_\mathrm{N}}\) for \(i=1,2,3\) and \(j=1,2,3\) using the normalized states.}
                    \par
                    \(i=1\), \(j=1\):
                    \begin{align*}
                        \prescript{}{\mathrm{N}}{\braket{\psi_1|\psi_1}_\mathrm{N}}&=1\quad\text{(by definition)}
                    \end{align*}
                    \(i=1\), \(j=2\):
                    \begin{align*}
                        \prescript{}{\mathrm{N}}{\braket{\psi_1|\psi_2}_\mathrm{N}}&=\left(\tfrac{1}{\sqrt{10}}\bra{\psi_1}\right)\left(\tfrac{1}{\sqrt 2}\ket{\psi_2}\right) \\
                        &=\tfrac{1}{\sqrt{20}}\braket{\psi_1|\psi_2} \\
                        &=\tfrac{1}{\sqrt{20}}(3\bra{+}+i\bra{-})\left(e^{i\pi/3}\ket{+}+\ket{-}\right) \\
                        &=\frac{1}{\sqrt{20}}\left(3e^{i\pi/3}\braket{+|+}+3\braket{+|-}+ie^{i\pi/3}\braket{-|+}+i\braket{-|-}\right) \\
                        &=\frac{1}{\sqrt{20}}\left(3e^{i\pi/3}+i\right)
                    \end{align*}
                    \(i=1\), \(j=3\):
                    \begin{align*}
                        \prescript{}{\mathrm{N}}{\braket{\psi_1|\psi_3}_\mathrm{N}}&=\left(\tfrac{1}{\sqrt{10}}\bra{\psi_1}\right)\left(\tfrac{1}{\sqrt{53}}\ket{\psi_3}\right) \\
                        &=\tfrac{1}{\sqrt{530}}\braket{\psi_1|\psi_3} \\
                        &=\tfrac{1}{\sqrt{530}}(3\bra{+}+i\bra{-})(7i\ket{+}-2\ket{-}) \\
                        &=\tfrac{1}{\sqrt{530}}(21i\braket{+|-}-6\braket{+|-}+7i^2\braket{-|+}-2i\braket{-|-}) \\
                        &=\tfrac{1}{\sqrt{530}}(21i-2i) \\
                        &=\frac{19i}{\sqrt{530}}
                    \end{align*}
                    \(i=2\), \(j=1\):
                    \begin{align*}
                        \prescript{}{\mathrm{N}}{\braket{\psi_2|\psi_1}_\mathrm{N}}&=\prescript{*}{\mathrm{N}}{\braket{\psi_1|\psi_2}_\mathrm{N}^*} \\
                        &=\conj\left(\frac{1}{\sqrt{20}}\left(3e^{i\pi/3}+i\right)\right) \\
                        &=\frac{1}{\sqrt{20}}\left(3e^{-i\pi/3}-i\right)
                    \end{align*}
                    \(i=2\), \(j=2\):
                    \begin{align*}
                        \prescript{}{\mathrm{N}}{\braket{\psi_2|\psi_2}_\mathrm{N}}&=1\quad\text{(by definition)}
                    \end{align*}
                    \(i=2\), \(j=3\):
                    \begin{align*}
                        \prescript{}{\mathrm{N}}{\braket{\psi_2|\psi_3}_\mathrm{N}}&=\left(\tfrac{1}{\sqrt{2}}\bra{\psi_2}\right)\left(\tfrac{1}{\sqrt{53}}\ket{\psi_3}\right) \\
                        &=\tfrac{1}{\sqrt{106}}\braket{\psi_2|\psi_3} \\
                        &=\frac{1}{\sqrt{106}}\left(e^{-i\pi/3}\bra{+}+\bra{-}\right)(7i\ket{+}-2\ket{-}) \\
                        &=\frac{1}{\sqrt{106}}\left(7ie^{-i\pi/3}\braket{+|+}-2e^{-i\pi/3}\braket{+|-}+7i\braket{-|+}-2\braket{-|-}\right) \\
                        &=\frac{1}{\sqrt{106}}\left(7e^{i\pi/6}-2\right)
                    \end{align*}
                    \(i=3\), \(j=1\):
                    \begin{align*}
                        \prescript{}{\mathrm{N}}{\braket{\psi_3|\psi_1}_\mathrm{N}}&=\prescript{*}{\mathrm{N}}{\braket{\psi_1|\psi_3}_\mathrm{N}^*} \\
                        &=\conj\left(\tfrac{19i}{\sqrt{530}}\right) \\
                        &=-\frac{19i}{\sqrt{530}}
                    \end{align*}
                    \(i=3\), \(j=2\):
                    \begin{align*}
                        \prescript{}{\mathrm{N}}{\braket{\psi_3|\psi_2}_\mathrm{N}}&=\prescript{*}{\mathrm{N}}{\braket{\psi_2|\psi_3}_\mathrm{N}^*} \\
                        &=\conj\left(\frac{1}{\sqrt{106}}\left(7e^{i\pi/6}-2\right)\right) \\
                        &=\frac{1}{\sqrt{106}}\left(7e^{-i\pi/6}-2\right)
                    \end{align*}
                    \(i=3\), \(j=3\):
                    \begin{align*}
                        \prescript{}{\mathrm{N}}{\braket{\psi_3|\psi_3}_\mathrm{N}}&=1\quad\text{(by definition)}
                    \end{align*}
                }
                \item{
                    \textbf{\boldmath For each state \(\ket{\psi_i}\), find the state \(\ket{\phi_i}\) with unit norm, \(\braket{\phi_i|\phi_i}=1\) that is orthogonal to it. Recall the orthogonality conditions for the basis states: \(\braket{+|+}=\braket{-|-}=1\) and \(\braket{+|-}=\braket{-|+}=0\).}
                    \par
                    If \(\ket{\psi_i}\) and \(\ket{\phi_i}\) are to be orthogonal, they must satisfy the orthogonality condition:
                    \[\braket{\phi_i|\psi_i}=0\]
                    Let's test this out with \(\ket{\psi_1}\) and \(\ket{\phi_1}\) to see if it works. First, we define \(\ket{\phi_1}\) to be some linear combination of the basis states:
                    \[\ket{\phi_1}=a\ket{+}+b\ket{-}\]
                    Then, we apply the orthogonality condition.
                    \begin{align*}
                        \braket{\phi_1|\psi_1}=0&=(a^*\bra{+}+b^*\bra{-})(3\ket{+}-i\ket{-}) \\
                        &=3a^*\braket{+|+}-b^*i\braket{-|-} \\
                        a^*&=\frac 1 3 b^*i \\
                        a&=-\frac 1 3 bi
                    \end{align*}
                    If this is correct, I should be able to pick any pair of \(a\) and \(b\) which satisfy this equation, and the \(\ket{\phi_1}\) they make should be orthogonal to \(\ket{\psi_1}\). I will randomly pick \(a=1\) and \(b=3i\), so
                    \[\ket{\phi_1}=\ket{+}+3i\ket{-}.\]
                    Now, we verify that this is orthogonal to \(\ket{\psi_1}\):
                    \begin{align*}
                        0&\stackrel{?}{=}\braket{\phi_1|\psi_1} \\
                        &\stackrel{?}{=}(\bra{+}-3i\bra{-})(3\ket{+}-i\ket{-}) \\
                        &\stackrel{?}{=}3\braket{+|+}+3i^2\braket{-|-} \\
                        &\stackrel{?}{=}3-3 \\
                        &=0
                    \end{align*}
                    Great! Now all that's left to do is normalize \(\ket{\phi_1}\) and we're done.
                    \begin{align*}
                        \braket{C\phi_1|C\phi_1}&=1 \\
                        1&=C^*(\bra{+}-3i\bra{-})\cdot C(\ket{+}+3i\ket{-}) \\
                        &=CC^*(\braket{+|+}-9i^2\braket{-|-}) \\
                        &=\abs{C}^2(1+9) \\
                        \abs{C}^2&=\frac{1}{10} \\
                        C&=\frac{1}{\sqrt{10}}
                    \end{align*}
                    \begin{equation*}
                        \ket{\phi_1}_\mathrm{N}=\tfrac{1}{\sqrt{10}}\ket{+}+\tfrac{3i}{\sqrt{10}}\ket{-}
                    \end{equation*}
                    Now I will repeat the process for finding \(\ket{\phi_2}\) and \(\ket{\phi_3}\). For \(\ket{\phi_2}\):
                    \begin{equation*}
                        \ket{\phi_2}=a_2\ket{+}+b_2\ket{-}
                    \end{equation*}
                    Applying the orthogonality condition:
                    \begin{align*}
                        \braket{\phi_2|\psi_2}=0&=(a_2^*\bra{+}+b_2^*\bra{-})\left(e^{i\pi/3}\ket{+}+\ket{-}\right) \\
                        &=a_2^*e^{i\pi/3}\braket{+|+}+b_2^*\braket{-|-} \\
                        a_2^*&=-e^{-i\pi/3}b_2^* \\
                        a_2&=-e^{i\pi/3}b_2
                    \end{align*}
                    Randomly picking \(a_2=-1\) and \(b_2=e^{-i\pi/3}\):
                    \begin{equation*}
                        \ket{\phi_2}=-\ket{+}+e^{-i\pi/3}\ket{-}
                    \end{equation*}
                    Normalizing:
                    \begin{align*}
                        \braket{C_2\phi_2|C_2\phi_2}&=1 \\
                        1&=C_2^*\left(-\bra{+}+e^{i\pi/3}\bra{-}\right)\cdot C_2\left(-\ket{+}+e^{-i\pi/3}\ket{-}\right) \\
                        &=C_2C_2^*(\braket{+|+}+0\braket{-|-}) \\
                        \abs{C_2}^2&=1 \\
                        C_2&=1
                    \end{align*}
                    \begin{equation*}
                        \ket{\phi_2}_\mathrm{N}=-\ket{+}+e^{-i\pi/3}\ket{-}
                    \end{equation*}
                    Finally, finding \(\ket{\phi_3}\):
                    \begin{equation*}
                        \ket{\phi_3}=a_3\ket{+}+b_3\ket{-}
                    \end{equation*}
                    Applying the orthogonality condition:
                    \begin{align*}
                        \braket{\phi_3|\psi_3}=0&=(a_3^*\bra{+}+b_3^*\bra{-})(7i\ket{+}-2\ket{-}) \\
                        &=7a_3^*i\braket{+|+}-2b_3^*\braket{-|-} \\
                        a_3^*&=\frac{2b_3^*}{7i} \\
                        &=-\tfrac 2 7 b_3^*i \\
                        a_3&=\tfrac 2 7 b_3i
                    \end{align*}
                    Randomly picking \(a_3=2\) and \(b_3=-7i\):
                    \begin{equation*}
                        \ket{\phi_3}=2\ket{+}-7i\ket{-}
                    \end{equation*}
                    Normalizing:
                    \begin{align*}
                        \braket{C_3\phi_3|C_3\phi_3}&=1 \\
                        1&=C_3^*(2\bra{+}+7i\bra{-})\cdot C_3(2\ket{+}-7i\ket{-}) \\
                        &=C_3C_3^*(4\braket{+|+}-49i^2\braket{-|-}) \\
                        \abs{C_3}^2&=\frac{1}{53} \\
                        C_3&=\frac{1}{\sqrt{53}}
                    \end{align*}
                    \begin{equation*}
                        \ket{\phi_3}_\mathrm{N}=\tfrac{2}{\sqrt{53}}\ket{+}-\tfrac{7i}{\sqrt{53}}\ket{-}
                    \end{equation*}
                }
                \item{
                    \textbf{\boldmath Postulate 4 of quantum mechanics tells us that the complex square of the inner product \(\abs{\braket{a|b}}^2\) is the probability of measuring a particular quantum state. For each of the normalized states \(\ket{\psi_i}_\mathrm{N}\), calculate the probability of measuring each of the six states indicated below. \[\ket{1}=\ket{+}\]}
                    With \(\ket{\psi_1}_\mathrm{N}\):
                    \begin{align*}
                        \abs{\braket{1|\psi_1}_\mathrm{N}}^2&=\Abs{\bra{+}\cdot\tfrac{1}{\sqrt{10}}(3\ket{+}-i\ket{-})}^2 \\
                        &=\Abs{\tfrac{3}{\sqrt{10}}\braket{+|+}-\tfrac{i}{\sqrt{10}}\braket{+|-}}^2 \\
                        &=\Abs{\tfrac{3}{\sqrt{10}}}^2 \\
                        &=\frac{9}{10}
                    \end{align*}
                    \par
                    With \(\ket{\psi_2}_\mathrm{N}\):
                    \begin{align*}
                        \abs{\braket{1|\psi_2}_\mathrm{N}}^2&=\Abs{\bra{+}\cdot\tfrac{1}{\sqrt 2}\left(e^{i\pi/3}\ket{+}+\ket{-}\right)}^2 \\
                        &=\Abs{\tfrac{1}{\sqrt 2}e^{i\pi/3}\braket{+|+}-\tfrac{1}{\sqrt 2}\braket{+|-}}^2 \\
                        &=\Abs{\tfrac{1}{\sqrt 2}e^{i\pi/3}}^2 \\
                        &=\frac{1}{2}
                    \end{align*}
                    \par
                    With \(\ket{\psi_3}_\mathrm{N}\):
                    \begin{align*}
                        \abs{\braket{1|\psi_3}_\mathrm{N}}^2&=\Abs{\bra{+}\cdot\tfrac{1}{\sqrt{53}}(7i\ket{+}-2\ket{-})}^2 \\
                        &=\Abs{\tfrac{7i}{\sqrt{53}}\braket{+|+}-\tfrac{2}{\sqrt{53}}\braket{+|-}}^2 \\
                        &=\Abs{\tfrac{7i}{\sqrt{53}}}^2 \\
                        &=\frac{49}{53}
                    \end{align*}
                    \par
                    {\boldmath \[\ket{2}=\ket{-}\]}
                    With \(\ket{\psi_1}_\mathrm{N}\):
                    \begin{align*}
                        \abs{\braket{2|\psi_1}_\mathrm{N}}^2&=\Abs{\bra{-}\cdot\tfrac{1}{\sqrt{10}}(3\ket{+}-i\ket{-})}^2 \\
                        &=\Abs{\tfrac{3}{\sqrt{10}}\braket{-|+}-\tfrac{i}{\sqrt{10}}\braket{-|-}}^2 \\
                        &=\Abs{-\tfrac{i}{\sqrt{10}}}^2 \\
                        &=\frac{1}{10}
                    \end{align*}
                    \par
                    With \(\ket{\psi_2}_\mathrm{N}\):
                    \begin{align*}
                        \abs{\braket{2|\psi_2}_\mathrm{N}}^2&=\Abs{\bra{-}\cdot\tfrac{1}{\sqrt 2}\left(e^{i\pi/3}\ket{+}+\ket{-}\right)}^2 \\
                        &=\Abs{\tfrac{1}{\sqrt 2}e^{i\pi/3}\braket{-|+}-\tfrac{1}{\sqrt 2}\braket{-|-}}^2 \\
                        &=\Abs{-\tfrac{1}{\sqrt 2}}^2 \\
                        &=\frac{1}{2}
                    \end{align*}
                    \par
                    With \(\ket{\psi_3}_\mathrm{N}\):
                    \begin{align*}
                        \abs{\braket{2|\psi_3}_\mathrm{N}}^2&=\Abs{\bra{-}\cdot\tfrac{1}{\sqrt{53}}(7i\ket{+}-2\ket{-})}^2 \\
                        &=\Abs{\tfrac{7i}{\sqrt{53}}\braket{-|+}-\tfrac{2}{\sqrt{53}}\braket{-|-}}^2 \\
                        &=\Abs{-\tfrac{2}{\sqrt{53}}}^2 \\
                        &=\frac{4}{53}
                    \end{align*}
                    \par
                    {\boldmath \[\ket{3}=\tfrac{1}{\sqrt 2}(\ket{+}+\ket{-})\]}
                    With \(\ket{\psi_1}_\mathrm{N}\):
                    \begin{align*}
                        \abs{\braket{3|\psi_1}_\mathrm{N}}^2&=\Abs{\tfrac{1}{\sqrt 2}(\bra{+}+\bra{-})\cdot\tfrac{1}{\sqrt{10}}(3\ket{+}-i\ket{-})}^2 \\
                        &=\Abs{\tfrac{1}{\sqrt{20}}(3\braket{+|+}-i\braket{-|-})}^2 \\
                        &=\Abs{\tfrac{1}{\sqrt{20}}(3-i)}^2 \\
                        &=\frac{10}{20} \\
                        &=\frac 1 2
                    \end{align*}
                    \par
                    With \(\ket{\psi_2}_\mathrm{N}\):
                    \begin{align*}
                        \abs{\braket{3|\psi_2}_\mathrm{N}}^2&=\Abs{\frac{1}{\sqrt 2}(\bra{+}+\bra{-})\cdot\frac{1}{\sqrt 2}\left(e^{i\pi/3}\ket{+}+\ket{-}\right)}^2 \\
                        &=\Abs{\frac 1 4 \left(e^{i\pi/3}\braket{+|+}+\braket{-|-}\right)}^2 \\
                        &=\Abs{\frac 1 4 \left(e^{i\pi/3}+1\right)}^2 \\
                        &=\frac{1}{16}\left(e^{i\pi/3}+1\right)\left(e^{-i\pi/3}+1\right) \\
                        &=\frac{1}{16}\left(1+e^{i\pi/3}+e^{-i\pi/3}+1\right) \\
                        &=\frac{1}{16}\left(2+2\cos\tfrac{\pi}{3}\right) \\
                        &=\frac{3}{16}
                    \end{align*}
                    \par
                    With \(\ket{\psi_3}_\mathrm{N}\):
                    \begin{align*}
                        \abs{\braket{3|\psi_3}_\mathrm{N}}^2&=\Abs{\tfrac{1}{\sqrt 2}(\bra{+}+\bra{-})\cdot\tfrac{1}{\sqrt{53}}(7i\ket{+}-2\ket{-})}^2 \\
                        &=\Abs{\tfrac{1}{\sqrt{106}}(7i\braket{+|+}-2\braket{-|-})}^2 \\
                        &=\Abs{\tfrac{1}{\sqrt{106}}(7i-2)}^2 \\
                        &=\frac{53}{106} \\
                        &=\frac 1 2
                    \end{align*}
                    \par
                    {\boldmath \[\ket{4}=\tfrac{1}{\sqrt 2}(\ket{+}-\ket{-})\]}
                    With \(\ket{\psi_1}_\mathrm{N}\):
                    \begin{align*}
                        \abs{\braket{4|\psi_1}_\mathrm{N}}^2&=\Abs{\tfrac{1}{\sqrt 2}(\bra{+}-\bra{-})\cdot\tfrac{1}{\sqrt{10}}(3\ket{+}-i\ket{-})}^2 \\
                        &=\Abs{\tfrac{1}{\sqrt{20}}(3\braket{+|+}+i\braket{-|-})}^2 \\
                        &=\Abs{\tfrac{1}{\sqrt{20}}(3+i)}^2 \\
                        &=\frac{10}{20} \\
                        &=\frac 1 2
                    \end{align*}
                    \par
                    With \(\ket{\psi_2}_\mathrm{N}\):
                    \begin{align*}
                        \abs{\braket{4|\psi_2}_\mathrm{N}}^2&=\Abs{\frac{1}{\sqrt 2}(\bra{+}-\bra{-})\cdot\frac{1}{\sqrt 2}\left(e^{i\pi/3}\ket{+}+\ket{-}\right)}^2 \\
                        &=\Abs{\frac 1 4 \left(e^{i\pi/3}\braket{+|+}-\braket{-|-}\right)}^2 \\
                        &=\Abs{\frac 1 4 \left(e^{i\pi/3}-1\right)}^2 \\
                        &=\frac{1}{16}\left(e^{i\pi/3}-1\right)\left(e^{-i\pi/3}-1\right) \\
                        &=\frac{1}{16}\left(1-e^{i\pi/3}-e^{-i\pi/3}+1\right) \\
                        &=\frac{1}{16}\left(2-2\cos\tfrac{\pi}{3}\right) \\
                        &=\frac{1}{16}
                    \end{align*}
                    \par
                    With \(\ket{\psi_3}_\mathrm{N}\):
                    \begin{align*}
                        \abs{\braket{4|\psi_3}_\mathrm{N}}^2&=\Abs{\tfrac{1}{\sqrt 2}(\bra{+}-\bra{-})\cdot\tfrac{1}{\sqrt{53}}(7i\ket{+}-2\ket{-})}^2 \\
                        &=\Abs{\tfrac{1}{\sqrt{106}}(7i\braket{+|+}+2\braket{-|-})}^2 \\
                        &=\Abs{\tfrac{1}{\sqrt{106}}(7i+2)}^2 \\
                        &=\frac{53}{106} \\
                        &=\frac 1 2
                    \end{align*}
                    \par
                    {\boldmath \[\ket{5}=\tfrac{1}{\sqrt 2}(\ket{+}+i\ket{-})\]}
                    With \(\ket{\psi_1}_\mathrm{N}\):
                    \begin{align*}
                        \abs{\braket{5|\psi_1}_\mathrm{N}}^2&=\Abs{\tfrac{1}{\sqrt 2}(\bra{+}+i\bra{-})\cdot\tfrac{1}{\sqrt{10}}(3\ket{+}-i\ket{-})}^2 \\
                        &=\Abs{\tfrac{1}{\sqrt{20}}(3\braket{+|+}-i^2\braket{-|-})}^2 \\
                        &=\Abs{\tfrac{1}{\sqrt{20}}(3+1)}^2 \\
                        &=\frac{16}{20} \\
                        &=\frac 4 5
                    \end{align*}
                    \par
                    With \(\ket{\psi_2}_\mathrm{N}\):
                    \begin{align*}
                        \abs{\braket{5|\psi_2}_\mathrm{N}}^2&=\Abs{\frac{1}{\sqrt 2}(\bra{+}+i\bra{-})\cdot\frac{1}{\sqrt 2}\left(e^{i\pi/3}\ket{+}+\ket{-}\right)}^2 \\
                        &=\Abs{\frac 1 4 \left(e^{i\pi/3}\braket{+|+}+i\braket{-|-}\right)}^2 \\
                        &=\Abs{\frac 1 4 \left(e^{i\pi/3}+i\right)}^2 \\
                        &=\frac{1}{16}\left(e^{i\pi/3}+i\right)\left(e^{-i\pi/3}-i\right) \\
                        &=\frac{1}{16}\left(1-ie^{i\pi/3}+ie^{-i\pi/3}-i^2\right) \\
                        &=\frac{1}{16}\left(2-i(e^{i\pi/3}-ie^{-i\pi/3})\right) \\
                        &=\frac{1}{16}\left(2-i\left(2i\sin\tfrac{\pi}{3}\right)\right) \\
                        &=\frac{1}{16}\left(2+\sqrt 3\right) \\
                        &=\frac{2+\sqrt 3}{16}
                    \end{align*}
                    \par
                    With \(\ket{\psi_3}_\mathrm{N}\):
                    \begin{align*}
                        \abs{\braket{5|\psi_3}_\mathrm{N}}^2&=\Abs{\tfrac{1}{\sqrt 2}(\bra{+}+i\bra{-})\cdot\tfrac{1}{\sqrt{53}}(7i\ket{+}-2\ket{-})}^2 \\
                        &=\Abs{\tfrac{1}{\sqrt{106}}(7i\braket{+|+}-2i\braket{-|-})}^2 \\
                        &=\Abs{\tfrac{1}{\sqrt{106}}(5i)}^2 \\
                        &=\frac{25}{106} \\
                        &=\frac 1 2
                    \end{align*}
                    \par
                    {\boldmath \[\ket{6}=\tfrac{1}{\sqrt 2}(\ket{+}-i\ket{-})\]}
                    With \(\ket{\psi_1}_\mathrm{N}\):
                    \begin{align*}
                        \abs{\braket{6|\psi_1}_\mathrm{N}}^2&=\Abs{\tfrac{1}{\sqrt 2}(\bra{+}-i\bra{-})\cdot\tfrac{1}{\sqrt{10}}(3\ket{+}-i\ket{-})}^2 \\
                        &=\Abs{\tfrac{1}{\sqrt{20}}(3\braket{+|+}+i^2\braket{-|-})}^2 \\
                        &=\Abs{\tfrac{1}{\sqrt{20}}(3-1)}^2 \\
                        &=\frac{4}{20} \\
                        &=\frac 1 5
                    \end{align*}
                    \par
                    With \(\ket{\psi_2}_\mathrm{N}\):
                    \begin{align*}
                        \abs{\braket{6|\psi_2}_\mathrm{N}}^2&=\Abs{\frac{1}{\sqrt 2}(\bra{+}-i\bra{-})\cdot\frac{1}{\sqrt 2}\left(e^{i\pi/3}\ket{+}+\ket{-}\right)}^2 \\
                        &=\Abs{\frac 1 4 \left(e^{i\pi/3}\braket{+|+}-i\braket{-|-}\right)}^2 \\
                        &=\Abs{\frac 1 4 \left(e^{i\pi/3}-i\right)}^2 \\
                        &=\frac{1}{16}\left(e^{i\pi/3}-i\right)\left(e^{-i\pi/3}+i\right) \\
                        &=\frac{1}{16}\left(1+ie^{i\pi/3}-ie^{-i\pi/3}-i^2\right) \\
                        &=\frac{1}{16}\left(2+i(e^{i\pi/3}-ie^{-i\pi/3})\right) \\
                        &=\frac{1}{16}\left(2+i\left(2i\sin\tfrac{\pi}{3}\right)\right) \\
                        &=\frac{1}{16}\left(2-\sqrt 3\right) \\
                        &=\frac{2-\sqrt 3}{16}
                    \end{align*}
                    \par
                    With \(\ket{\psi_3}_\mathrm{N}\):
                    \begin{align*}
                        \abs{\braket{6|\psi_3}_\mathrm{N}}^2&=\Abs{\tfrac{1}{\sqrt 2}(\bra{+}-i\bra{-})\cdot\tfrac{1}{\sqrt{53}}(7i\ket{+}-2\ket{-})}^2 \\
                        &=\Abs{\tfrac{1}{\sqrt{106}}(7i\braket{+|+}+2i\braket{-|-})}^2 \\
                        &=\Abs{\tfrac{1}{\sqrt{106}}(9i)}^2 \\
                        &=\frac{81}{106}
                    \end{align*}
                }
            \end{enumerate}
        }
    \end{enumerate}
\end{document}
