\documentclass[11pt]{article}

\usepackage{amsmath}
\usepackage{booktabs}
\usepackage{chemfig}
\usepackage{chemformula}
\usepackage{enumitem}
\usepackage[T1]{fontenc}
\usepackage[margin=1in]{geometry}
\usepackage{graphicx}
\usepackage[utf8]{inputenc}
\usepackage{libertine}
\usepackage{mathtools}
\usepackage[libertine]{newtxmath}
\usepackage{pgfplots}
\usepackage[detect-weight=true]{siunitx}
\usepackage{suffix}

\title{MATH 228 Assignment 1}
\author{Brandon Tsang}
\date{June 10, 2020}

\pgfplotsset{compat=1.16}
\setlength{\parindent}{0pt}
\DeclareMathOperator{\diff}{d}
\newcommand\dd[2]{\frac{\diff #1}{\diff #2}}
\WithSuffix\newcommand\dd*[1]{\frac{\diff{}}{\diff #1}\,}
\DeclarePairedDelimiter\abs{\lvert}{\rvert}

\begin{document}
    \maketitle
    \begin{enumerate}[label=\textbf{\arabic*.}]
        \item{
            \textbf{\boldmath Find the solution of the initial value problem \[y'-y=2te^t,\quad y(0)=1.\]}%
            If we multiply both sides by some function \(\mu=\mu(t)\), we get
            \begin{equation}
                \label{eqn:1}
                \mu y'-\mu y=2\mu te^t.
            \end{equation}
            And if we choose a \(\mu\) such that \[\dd{\mu}{t}=-\mu,\] the left side of equation (\ref{eqn:1}) will be equal to \(\dd*{t}\mu y\).
            \begin{align*}
                \dd{\mu}{t}&=-\mu \\
                \frac 1 \mu \diff\mu&=-\diff t \\
                \ln\abs{\mu}&=-t+C_1 \\
                \abs\mu&=e^{-t}e^{C_1} \\
                \mu&=ce^{-t}
            \end{align*}
            We can discard the constant \(c\) since we don't need the most general expression for \(\mu\). Next, we substitute \(\dd*{t}\mu y\) in for \(\mu y'-\mu y\) in equation (\ref{eqn:1}).
            \begin{align*}
                \dd*{t}\mu y&=2\mu te^t \\
                \dd*{t}e^{-t}y&=2t \\
                e^{-t}y&=t^2+C \\
                \Aboxed{y&=e^tt^2+Ce^t}
            \end{align*}
        }
    \end{enumerate}
\end{document}
