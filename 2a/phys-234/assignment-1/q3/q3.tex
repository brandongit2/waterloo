\documentclass[11pt]{article}

\usepackage{amsmath}
\usepackage{braket}
\usepackage{booktabs}
\usepackage[outline]{contour}
\usepackage{enumitem}
\usepackage[T1]{fontenc}
\usepackage[margin=1in]{geometry}
\usepackage{graphicx}
\usepackage[utf8]{inputenc}
\usepackage{libertine}
\usepackage{mathtools}
\usepackage[libertine]{newtxmath}
\usepackage{parskip}
\usepackage[detect-weight=true]{siunitx}

\title{PHYS 234 Assignment 1}
\author{Brandon Tsang}
\date{May 18, 2020}

\contourlength{0.1em}

\DeclareMathOperator\conj{conj}
\newcommand\abs[1]{\lvert#1\rvert}
\newcommand\Abs[1]{\left\lvert#1\right\rvert}
\newcommand\uline[1]{\underline{\smash{#1}}\llap{\contour{white}{#1}}}

\newenvironment{amatrix}[1]{%
    \left[\begin{array}{@{}*{#1}{c}|c@{}}
}{%
    \end{array}\right]
}

\begin{document}
    \maketitle
    \begin{enumerate}[label=\textbf{\arabic*.}, start=3]
        \item{
            \textbf{\uline{Eigenvalues and Eigenvectors} \\ Find the eigenvalues and eigenvectors of the following matrices:}
            \begin{enumerate}[label=\textbf{(\alph*)}]
                \item{
                    \textbf{\boldmath \(\begin{bmatrix}0 & 1 \\ 1 & 0\end{bmatrix}\)}
                    \par
                    If the matrix is represented by \(A\), then
                    \[(A-\lambda I)\mathbf{v}=0\]
                    where \(I\) is the identity matrix with dimensions of \(A\) and \(\lambda\) represents the eigenvalues. Solving this equation:
                    \begin{align*}
                        \left(\begin{bmatrix}0 & 1 \\ 1 & 0\end{bmatrix}-\begin{bmatrix}\lambda & 0 \\ 0 & \lambda\end{bmatrix}\right)\mathbf{v}&=0 \\
                        \begin{bmatrix}-\lambda & 1 \\ 1 & -\lambda\end{bmatrix}\mathbf{v}&=0 \\
                        \begin{vmatrix}-\lambda & 1 \\ 1 & -\lambda\end{vmatrix}&=0 \\
                        (-\lambda)(-\lambda)-(1)(1)&=0 \\
                        \lambda^2-1&=0 \\
                        \lambda&=\pm 1
                    \end{align*}
                    These are the eigenvalues. To find their associated eigenvectors, we substitute them into the original equation. For \(\lambda=-1\):
                    \begin{align*}
                        \left(\begin{bmatrix}0 & 1 \\ 1 & 0\end{bmatrix}-\begin{bmatrix}-1 & 0 \\ 0 & -1\end{bmatrix}\right)\mathbf{v}&=0 \\
                        \begin{bmatrix}1 & 1 \\ 1 & 1\end{bmatrix}\mathbf{v}&=0
                    \end{align*}
                    \begin{equation*}
                        \begin{amatrix}{2}1 & 1 & 0 \\ 1 & 1 & 0\end{amatrix}
                        \xrightarrow{R_2-R_1}
                        \begin{amatrix}{2}1 & 1 & 0 \\ 0 & 0 & 0\end{amatrix}
                    \end{equation*}
                    If \(\mathbf{v}=\begin{bmatrix}v_1 \\ v_2\end{bmatrix}\), the above matrix corresponds to a solution of \(v_1+v_2=0\), or \(v_1=-v_2\). Therefore,
                    \[\mathbf{v}=\begin{bmatrix}-v_2 \\ v_2\end{bmatrix}=v_2\begin{bmatrix}-1 \\ 1\end{bmatrix}\]
                    and \(\begin{bmatrix}-1 \\ 1\end{bmatrix}\) is the eigenvector for \(\lambda=-1\).
                    \par
                    For \(\lambda=1\):
                    \begin{align*}
                    \left(\begin{bmatrix}0 & 1 \\ 1 & 0\end{bmatrix}-\begin{bmatrix}1 & 0 \\ 0 & 1\end{bmatrix}\right)\mathbf{v}&=0 \\
                    \begin{bmatrix}-1 & 1 \\ 1 & -1\end{bmatrix}\mathbf{v}&=0
                    \end{align*}
                    \begin{equation*}
                        \begin{amatrix}{2}-1 & 1 & 0 \\ 1 & -1 & 0\end{amatrix}
                        \xrightarrow{R_2+R_1}
                        \begin{amatrix}{2}-1 & 1 & 0 \\ 0 & 0 & 0\end{amatrix}
                    \end{equation*}
                    If \(\mathbf{v}=\begin{bmatrix}v_1 \\ v_2\end{bmatrix}\), the above matrix corresponds to a solution of \(-v_1+v_2=0\), or \(v_1=v_2\). Therefore,
                    \[\mathbf{v}=\begin{bmatrix}v_2 \\ v_2\end{bmatrix}=v_2\begin{bmatrix}1 \\ 1\end{bmatrix}\]
                    and \(\begin{bmatrix}1 \\ 1\end{bmatrix}\) is the eigenvector for \(\lambda=1\).
                    \par
                    This same procedure will be followed for the rest of this question.
                }
                \item{
                    \textbf{\boldmath \(\begin{bmatrix}4 & 1 \\ 1 & -2\end{bmatrix}\)}
                    \par
                    To find the eigenvalues:
                    \begin{align*}
                        (A-\lambda I)\mathbf{v}&=0 \\
                        \left(\begin{bmatrix}4 & 1 \\ 1 & -2\end{bmatrix}-\begin{bmatrix}\lambda & 0 \\ 0 & \lambda\end{bmatrix}\right)\mathbf{v}&=0 \\
                        \begin{bmatrix}4-\lambda & 1 \\ 1 & -2-\lambda\end{bmatrix}\mathbf{v}&=0 \\
                        \begin{vmatrix}4-\lambda & 1 \\ 1 & -2-\lambda\end{vmatrix}&=0 \\
                        (4-\lambda)(-2-\lambda)-(1)(1)&=0 \\
                        -8-4\lambda+2\lambda+\lambda^2-1&=0 \\
                        \lambda^2-2\lambda-9&=0 \\
                        \lambda&=1\pm\sqrt{10}
                    \end{align*}
                    Then, finding the eigenvector for \(\lambda=1-\sqrt{10}\):
                    \begin{align*}
                        (A-\lambda I)\mathbf{v}&=0 \\
                        \left(\begin{bmatrix}4 & 1 \\ 1 & -2\end{bmatrix}-\begin{bmatrix}1-\sqrt{10} & 0 \\ 0 & 1-\sqrt{10}\end{bmatrix}\right)\mathbf{v}&=0 \\
                        \begin{bmatrix}3+\sqrt{10} & 1 \\ 1 & \sqrt{10}-3\end{bmatrix}\mathbf{v}&=0
                    \end{align*}
                    \begin{equation*}
                        \begin{amatrix}{2}3+\sqrt{10} & 1 & 0 \\ 1 & \sqrt{10}-3 & 0\end{amatrix}
                        \xrightarrow{R_1-\left(3+\sqrt{10}\right)R_2}
                        \begin{amatrix}{2}0 & 0 & 0 \\ 1 & \sqrt{10}-3 & 0\end{amatrix}
                        \xrightarrow{R_1\leftrightarrow R_2}
                        \begin{amatrix}{2}1 & \sqrt{10}-3 & 0 \\ 0 & 0 & 0\end{amatrix}
                    \end{equation*}
                    If \(\mathbf{v}=\begin{bmatrix}v_1 \\ v_2\end{bmatrix}\), the above matrix corresponds to a solution of \(v_1+\left(\sqrt{10}-3\right)v_2=0\), or \(v_1=\left(3-\sqrt{10}\right)v_2\). Therefore,
                    \[\mathbf{v}=\begin{bmatrix}\left(3-\sqrt{10}\right)v_2 \\ v_2\end{bmatrix}=v_2\begin{bmatrix}3-\sqrt{10} \\ 1\end{bmatrix}\]
                    so the eigenvector for \(\lambda=1-\sqrt{10}\) is \(\begin{bmatrix}3-\sqrt{10} \\ 1\end{bmatrix}\).
                    \par
                    Finally, finding the eigenvector for \(\lambda=1+\sqrt{10}\):
                    \begin{align*}
                        (A-\lambda I)\mathrm{v}&=0 \\
                        \left(\begin{bmatrix}4 & 1 \\ 1 & -2\end{bmatrix}-\begin{bmatrix}1+\sqrt{10} & 0 \\ 0 & 1+\sqrt{10}\end{bmatrix}\right)\mathbf{v}&=0 \\
                        \begin{bmatrix}3-\sqrt{10} & 1 \\ 1 & -3-\sqrt{10}\end{bmatrix}\mathbf{v}&=0
                    \end{align*}
                    \begin{equation*}
                        \begin{amatrix}{2}3-\sqrt{10} & 1 & 0 \\ 1 & -3-\sqrt{10} & 0\end{amatrix}
                        \xrightarrow{R_1-\left(3-\sqrt{10}\right)R_2}
                        \begin{amatrix}{2}0 & 0 & 0 \\ 1 & -3-\sqrt{10} & 0\end{amatrix}
                        \xrightarrow{R_1\leftrightarrow R_2}
                        \begin{amatrix}{2}1 & -3-\sqrt{10} & 0 \\ 0 & 0 & 0\end{amatrix}
                    \end{equation*}
                    If \(\mathbf{v}=\begin{bmatrix}v_1 \\ v_2\end{bmatrix}\), the above matrix corresponds to a solution of \(v_1+\left(-3-\sqrt{10}\right)v_2=0\), or \(v_1=\left(3+\sqrt{10}\right)v_2\). Therefore,
                    \[\mathbf{v}=\begin{bmatrix}\left(3+\sqrt{10}\right)v_2 \\ v_2\end{bmatrix}=v_2\begin{bmatrix}3+\sqrt{10} \\ 1\end{bmatrix}\]
                    so the eigenvector for \(\lambda=1+\sqrt{10}\) is \(\begin{bmatrix}3+\sqrt{10} \\ 1\end{bmatrix}\).
                }
                \item{
                    \textbf{\boldmath \(\begin{bmatrix}0 & -i \\ i & 0\end{bmatrix}\)}
                    \textit{Sorry, incomplete.}
                }
            \end{enumerate}
        }
    \end{enumerate}
\end{document}
