\documentclass[11pt]{article}

\usepackage{amsmath}
\usepackage{booktabs}
\usepackage{braket}
\usepackage{chemfig}
\usepackage{chemformula}
\usepackage[outline]{contour}
\usepackage{enumitem}
\usepackage[T1]{fontenc}
\usepackage[margin=1in]{geometry}
\usepackage{graphicx}
\usepackage[utf8]{inputenc}
\usepackage{libertine}
\usepackage{mathtools}
\usepackage[libertine]{newtxmath}
\usepackage{pgfplots}
\usepackage[detect-weight=true]{siunitx}

\title{PHYS 234 Assignment 2}
\author{Brandon Tsang}
\date{May 29, 2020}

\pgfplotsset{compat=1.17}
\contourlength{0.1em}
\newcommand\uline[1]{\underline{\smash{#1}}\llap{\contour{white}{#1}}}
\DeclarePairedDelimiter\abs{\lvert}{\rvert}
\makeatletter
\let\oldabs\abs
\def\abs{\@ifstar{\oldabs}{\oldabs*}}
\let\oldnorm\norm
\def\norm{\@ifstar{\oldnorm}{\oldnorm*}}
\makeatother

\newenvironment{amatrix}[1]{%
    \left[\begin{array}{@{}*{#1}{c}|c@{}}
}{%
    \end{array}\right]
}

\begin{document}
    \maketitle
    \begin{enumerate}[label=\textbf{\arabic*.}, start=4]
        \item{
            \textbf{\boldmath \uline{State Tomography} \\ It is known that there is a 90\% probability of obtaining \(S_z=\frac \hbar 2\) if a measurement of \(S_z\) is carried out on a spin-\(\frac 1 2\) particle. In addition, it is known that there is a 20\% probability of obtaining \(S_y=\frac \hbar 2\) if a measurement of \(S_y\) is carried out. Determine the spin state of a particle as completely as possible from this information. What is the probability of obtaining \(S_x=-\frac \hbar 2\) if a measurement of \(S_x\) is carried out?}
            \par
            From the first criterion:
            \begin{align*}
                \abs{\braket{+|\psi}}^2&=\tfrac{9}{10} \\
                \abs{\braket{+|\psi}}&=\tfrac{3}{\sqrt{10}}
            \end{align*}
            If \(\ket\psi=a\ket{+}+b\ket{-}\):
            \begin{align*}
                \abs{\bra{+}(a\ket{+}+b\ket{-})}&=\tfrac{3}{\sqrt{10}} \\
                \abs{a}&=\tfrac{3}{\sqrt{10}} \\
                a&=\frac{3}{\sqrt{10}}e^{i\alpha}
            \end{align*}
            And from the second criterion: 
            \begin{align*}
                \abs{\prescript{}{y}{\braket{+|\psi}}}^2&=\frac 1 5 \\
                \abs{\prescript{}{y}{\braket{+|\psi}}}^2&=\frac 1 5 \\
                \abs{\tfrac{1}{\sqrt 2}(\bra{+}-i\bra{-})(a\ket{+}+b\ket{-})}^2&=\frac 1 5 \\
                \abs{\tfrac{1}{\sqrt 2}(a-bi)}^2&=\frac 1 5 \\
                \frac 1 2 (\abs{a}^2+\abs{b}^2+ab^*i-a^*bi)&=\frac 1 5 \\
                1+ab^*i-a^*bi&=\frac 2 5 \\
                a\abs{b}e^{-i\beta}i-a^*\abs{b}e^{i\beta}i&=-\frac 3 5 \\
                \abs{b}\left(ae^{-i\beta}-a^*e^{i\beta}\right)&=\frac{3i}{5} \\
                \abs{a}\abs{b}\left(e^{i\alpha}e^{-i\beta}-e^{-i\alpha}e^{i\beta}\right)&=\frac{3i}{5} \\
                \abs{a}\abs{b}\left(e^{i(\alpha-\beta)}-e^{i(\beta-\alpha)}\right)&=\frac{3i}{5}
            \end{align*}
            Arbitrarily set \(\beta=0\):
            \begin{align*}
                \abs{a}\abs{b}\left(e^{i\alpha}-e^{-i\alpha}\right)&=\frac{3i}{5} \\
                \abs{b}\left(e^{i\alpha}-e^{-i\alpha}\right)&=\frac{3\sqrt{10}i}{15} \\
                b&=\frac{\sqrt{10}i}{5\left(e^{i\alpha}-e^{-i\alpha}\right)}
            \end{align*}
            So \(\ket\psi=\frac{3}{\sqrt{10}}e^{i\alpha}\ket{+}+\frac{\sqrt{10}i}{5\left(e^{i\alpha}-e^{-i\alpha}\right)}\ket{-}\).
            \par
            Now, we find the probability of measuring \(S_x=-\frac \hbar 2\):
            \begin{align*}
                \abs{\prescript{}{x}{\braket{-|\psi}}}^2&=\abs{\frac{1}{\sqrt 2}(\bra{+}-\bra{-})\left(\frac{3}{\sqrt{10}}e^{i\alpha}\ket{+}+\frac{\sqrt{10}i}{5\left(e^{i\alpha}-e^{-i\alpha}\right)}\ket{-}\right)}^2 \\
                &=\abs{\frac{1}{\sqrt 2}\left(\frac{3}{\sqrt{10}}e^{i\alpha}-\frac{\sqrt{10}i}{5\left(e^{i\alpha}-e^{-i\alpha}\right)}\right)}^2 \\
            \end{align*}
        }
    \end{enumerate}
\end{document}
