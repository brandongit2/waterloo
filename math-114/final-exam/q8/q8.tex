\documentclass[11pt]{article}

\usepackage{amsmath}
\usepackage{booktabs}
\usepackage{enumitem}
\usepackage[T1]{fontenc}
\usepackage[margin=1in]{geometry}
\usepackage{graphicx}
\usepackage[utf8]{inputenc}
\usepackage{libertine}
\usepackage[libertine]{newtxmath}

\title{MATH 114 Final Exam Question 8}
\author{Brandon Tsang}
\date{April 14, 2020}

\begin{document}
    \maketitle
    \begin{enumerate}[label=\textbf{\arabic*.}, start=8]
        \item{
            \textbf{\boldmath Imagine I give you some information about two vectors \(\vec{\mathrm a}\) and \(\vec{\mathrm b}\), that have the same number of components. You know the values of \(a_1\), \(a_3\), \(a_5\), \(b_2\), \(b_3\), \(b_4\), and \(b_5\). But you do not know any of the other values in the vectors. That is:
            \[\begin{array}{r @{{}=[} c @{,{}} c @{,{}} c @{,{}} c @{,{}} c @{,{}} c @{,{}} c @{]^T}}
                \vec{\mathrm a}&a_1&?  &a_3&?  &a_5&\ldots&? \\
                \vec{\mathrm b}&?  &b_2&b_3&b_4&b_5&\ldots&?
            \end{array}\]
            Describe how you can create a vector \(\vec{\mathrm v}\), such that \(\vec{\mathrm v}\neq\vec{0}\) and \(\vec{\mathrm v}\) is orthogonal to both \(\vec{\mathrm a}\) and \(\vec{\mathrm b}\).}
            \par
        }
    \end{enumerate}
\end{document}
