\documentclass[11pt]{article}

\usepackage{amsmath}
\usepackage{booktabs}
\usepackage{enumitem}
\usepackage[T1]{fontenc}
\usepackage[margin=1in]{geometry}
\usepackage{graphicx}
\usepackage[utf8]{inputenc}
\usepackage{libertine}
\usepackage[libertine]{newtxmath}

\title{MATH 114 Assignment 9, Q2}
\author{Brandon Tsang}
\date{April 3, 2020}

\newenvironment{amatrix}[1]{%
    \left[\begin{array}{@{}*{#1}{c}|c@{}}
}{%
    \end{array}\right]
}

\begin{document}
    \maketitle
    \begin{enumerate}[label=\textbf{\arabic*.}]
        \setcounter{enumi}{1}
        \item{
            \textbf{\boldmath In class we have been working only with $2\times2$ matrices. Partly this is because we usually can't easily find the roots of polynomials of larger degree. (Octave and Matlab will be using approximation techniques to solve them.)}
            \par
            \textbf{\boldmath Consider the matrix $$\begin{bmatrix}1 & 0 & 1 \\ 1 & 3 & 4 \\ 2 & 2 & 4\end{bmatrix}.$$}
            \begin{enumerate}[label=\textbf{(\alph*)}]
                \item{
                    \textbf{\boldmath Find the eigenvalues. (You should find that $\lambda=0$ is a solution; then what remains is a quadratic, and that gives you the other two.)}
                    \begin{align*}
                        \begin{bmatrix}
                            1 & 0 & 1 \\
                            1 & 3 & 4 \\
                            2 & 2 & 4
                        \end{bmatrix}
                        \mathbf{u}
                        &=
                        \lambda\mathbf{u} \\
                        \begin{bmatrix}
                            1 & 0 & 1 \\
                            1 & 3 & 4 \\
                            2 & 2 & 4
                        \end{bmatrix}
                        \mathbf{u}-\lambda\mathbf{u}&=0 \\
                        \left(
                        \begin{bmatrix}
                            1 & 0 & 1 \\
                            1 & 3 & 4 \\
                            2 & 2 & 4
                        \end{bmatrix}
                        -\lambda I_3\right)\mathbf{u}&=0 \\
                        \det\left(
                        \begin{bmatrix}
                            1 & 0 & 1 \\
                            1 & 3 & 4 \\
                            2 & 2 & 4
                        \end{bmatrix}
                        -\lambda I_3\right)
                        &=0 \\
                        \begin{vmatrix}
                            1-\lambda & 0 & 1 \\
                            1 & 3-\lambda & 4 \\
                            2 & 2 & 4-\lambda
                        \end{vmatrix}
                        &=0 \\
                        (1-\lambda)
                        \begin{vmatrix}
                            3-\lambda & 4 \\
                            2 & 4-\lambda
                        \end{vmatrix}
                        -0
                        \begin{vmatrix}
                            1 & 4 \\
                            2 & 4-\lambda
                        \end{vmatrix}
                        +1
                        \begin{vmatrix}
                            1 & 3-\lambda \\
                            2 & 2
                        \end{vmatrix}
                        &=0 \\
                        (1-\lambda)(3-\lambda)(4-\lambda)-(1-\lambda)(4)(2)-0+(1)(2)-(3-\lambda)(2)&=0 \\
                        (\lambda^2-4\lambda+3)(4-\lambda)-(8-8\lambda)+2-(6-2\lambda)&=0 \\
                        -\lambda^3+8\lambda^2-19\lambda+12-8+8\lambda+2-6+2\lambda&=0 \\
                        -\lambda^3+8\lambda^2-9\lambda&=0 \\
                        \lambda(\lambda^2-8\lambda+9)&=0
                    \end{align*}
                    One solution is $\lambda=0$; the other two will need the quadratic formula:
                    \begin{align*}
                        \lambda&=\frac{-b\pm\sqrt{b^2-4ac}}{2a} \\
                        &=\frac{8\pm\sqrt{(-8)^2-4(1)(9)}}{2(1)} \\
                        &=\frac{8\pm\sqrt{28}}{2} \\
                        &=4\pm\sqrt{7}
                    \end{align*}
                    So the eigenvalues are $\lambda=0$, $\lambda=4+\sqrt{7}$, and $\lambda=4-\sqrt{7}$.
                }
                \item{
                    \textbf{\boldmath Find the eigenvector for $\lambda=0$.}
                    \begin{align*}
                        \begin{bmatrix}
                            1 & 0 & 1 \\
                            1 & 3 & 4 \\
                            2 & 2 & 4
                        \end{bmatrix}
                        \mathbf{u}
                        &=
                        0\mathbf{u} \\
                        \begin{bmatrix}
                            1 & 0 & 1 \\
                            1 & 3 & 4 \\
                            2 & 2 & 4
                        \end{bmatrix}
                        \mathbf{u}&=\mathbf{0}
                    \end{align*}
                    Setting up an augmented matrix:
                    \begin{gather*}
                        \begin{amatrix}{3}
                            1 & 0 & 1 & 0 \\
                            1 & 3 & 4 & 0 \\
                            2 & 2 & 4 & 0
                        \end{amatrix}
                        \sim
                        \begin{amatrix}{3}
                            1 & 0 & 1 & 0 \\
                            0 & 3 & 3 & 0 \\
                            0 & 2 & 2 & 0
                        \end{amatrix}
                        \sim
                        \begin{amatrix}{3}
                            1 & 0 & 1 & 0 \\
                            0 & 1 & 1 & 0 \\
                            0 & 2 & 2 & 0
                        \end{amatrix}
                        \sim
                        \begin{amatrix}{3}
                            1 & 0 & 1 & 0 \\
                            0 & 1 & 1 & 0 \\
                            0 & 0 & 0 & 0
                        \end{amatrix}
                    \end{gather*}
                    This represents the system
                    \begin{align*}
                            u_1+u_3&=0 \\
                            u_2+u_3&=0 \\
                            0u_3&=0
                    \end{align*}
                    where \(u_3\) is a free variable, so it will be called \(s\). This gives
                    \begin{align*}
                        u_1&=-s \\
                        u_2&=-s \\
                        u_3&=s
                    \end{align*}
                    Finally, we get:
                    \begin{equation*}
                        \mathbf{u}=s
                        \begin{bmatrix}
                            -1 \\ -1 \\ 1
                        \end{bmatrix}
                    \end{equation*} 
                    Therefore, the eigenvector is \(\begin{bmatrix}-1 \\ -1 \\ 1\end{bmatrix}\).
                }
            \end{enumerate}
        }
    \end{enumerate}
\end{document}
