\documentclass[11pt]{article}

\usepackage{amsmath}
\usepackage{booktabs}
\usepackage{enumitem}
\usepackage[T1]{fontenc}
\usepackage[margin=1in]{geometry}
\usepackage{graphicx}
\usepackage[utf8]{inputenc}
\usepackage{libertine}
\usepackage[libertine]{newtxmath}
\usepackage{siunitx}

\title{PHYS 124 Final Exam Question 1}
\author{Brandon Tsang}
\date{April 23, 2020}

\begin{document}
    \maketitle
    \begin{enumerate}[label=\textbf{\arabic*.}]
        \item{
            \begin{enumerate}[label=\textbf{(\alph*)}]
                \item{
                    \textbf{Which processes below are physically forbidden? Why?}
                    \begin{enumerate}[label=\textbf{(\roman*)}]
                        \item{
                            \textbf{\boldmath \(\tau^++\tau^-\rightarrow\mu^++e^-\)}
                            \par
                            On the left side,
                            \begin{align*}
                                L_\tau&=-1+1=0 \\
                                L_\mu&=0 \\
                                L_\mathrm{e}&=0.
                            \end{align*}
                            On the right side,
                            \begin{align*}
                                L_\tau&=0 \\
                                L_\mu&=-1 \\
                                L_\mathrm{e}&=1.
                            \end{align*}
                            The conservation of \(L_\mu\) and \(L_\mathrm{e}\) are violated, so this is forbidden.
                        }
                        \item{
                            \textbf{\boldmath \(\mathrm{p}+\bar{\mathrm{n}}\rightarrow\mathrm{p}+\mathrm{n}\)}
                            \par
                            On the left side, \(B=1-1=0\). On the right side, \(B=1+1=2\). The conservation of baryon number is violated, so this is forbidden.
                        }
                        \item{
                            \textbf{\boldmath \(\mathrm{u}+\bar{\mathrm{u}}\rightarrow\bar{\mathrm{s}}+\mathrm{s}\)}
                            \par
                            On the left side,
                            \begin{align*}
                                Q/e&=\frac{2}{3}-\frac{2}{3}=0 \\
                                B&=\frac{1}{3}-\frac{1}{3}=0 \\
                                S&=0.
                            \end{align*}
                            On the right side,
                            \begin{align*}
                                Q/e&=\frac{1}{3}-\frac{1}{3}=0 \\
                                B&=-\frac{1}{3}+\frac{1}{3}=0 \\
                                S&=1-1=0.
                            \end{align*}
                            Charge, baryon number, and strangeness are all conserved, so this is allowed.
                        }
                        \item{
                            \textbf{\boldmath \(\gamma+\gamma\rightarrow\mu^++\mu^-\)}
                            \par
                            On the left side, \(L_\mu=0\). On the right side, \(L_\mu=-1+1=0\). \(L_\mu\) is conserved, so this is allowed.
                        }
                        \item{
                            \textbf{\boldmath \(\mathrm{n}+\pi^-\rightarrow\bar{\mathrm{p}}+\Lambda\)}
                            \par
                            On the left side,
                            \begin{align*}
                                Q/e&=0-1=-1 \\
                                B&=1 \\
                                S&=0.
                            \end{align*}
                            On the right side,
                            \begin{align*}
                                Q/e&=-1+0=-1 \\
                                B&=-1+1=0 \\
                                S&=-1.
                            \end{align*}
                            Charge, baryon number, and strangeness are all violated, so this is forbidden.
                        }
                        \item{
                            \textbf{\boldmath \(\mathrm{p}+\bar{\mathrm{p}}\rightarrow\gamma+\gamma+\gamma\)}
                            \par
                            On the left side, \(B=1-1=0\). On the right side, \(B=0\). Baryon number is conserved, so this is allowed.
                        }
                    \end{enumerate}
                }
                \item{
                    \textbf{The Future Circular Collider (FCC), if built, will collide two beams of protons, each beam having an average energy of 50 TeV. How much energy would a beam in a fixed-target experiment have to have to get the same average total amount of useable energy per collision?}
                    \par
                    The useable energy from the collision of a particle into a stationary particle with similar mass is given by \[E_\mathrm{a}^2=2mc^2(E_m+mc^2)\]
                    where \(m\) is the mass of each particle and \(E_m\) is the total energy of the moving particle. Solving this for \(E_m\):
                    \begin{align*}
                        E_\mathrm{a}^2&=2mc^2(E_m+mc^2) \\
                        E_m+mc^2&=\frac{E_\mathrm{a}^2}{2mc^2} \\
                        E_m&=\frac{E_\mathrm{a}^2}{2mc^2}-mc^2 \\
                        &=\tfrac{(\SI{8.01088e-6}{\joule})^2}{2(\SI{1.67262e-27}{\kilogram})(\SI{2.99792e8}{\meter\per\second})^2}-(\SI{1.67262e-27}{\kilogram})(\SI{2.99792e8}{\meter\per\second})^2 \\
                        &=\SI{0.213448}{\joule}
                    \end{align*}
                }
                \item{
                    \textbf{\boldmath An experiment is performed at Fermilab to find a new particle \(X\) from the scattering process \[\mathrm{p}+\mathrm{p}\rightarrow K^++K^++X.\] What are the values of the electric charge, strangeness, and baryon number of the \(X\) particle? What quarks must it be made of?}
                    \par
                    On the left side, 
                    \begin{align*}
                        Q/e&=0 \\
                        B&=1+1=2 \\
                        S&=0.
                    \end{align*}On the right side,
                    \begin{align*}
                        Q/e&=1+1+Q_X/e \\
                        B&=0+B_X \\
                        S&=S_X.
                    \end{align*}
                    To abide by the conservation laws, \(Q_X/e\) must be -2, \(B_X\) must be 2, and \(S_X\) must be 0.
                }
            \end{enumerate}
        }
    \end{enumerate}
\end{document}
