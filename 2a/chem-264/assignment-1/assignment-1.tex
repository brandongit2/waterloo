\documentclass[11pt]{article}

\usepackage{amsmath}
\usepackage{booktabs}
\usepackage{chemfig}
\usepackage{chemformula}
\usepackage{enumitem}
\usepackage[T1]{fontenc}
\usepackage[margin=1in]{geometry}
\usepackage{graphicx}
\usepackage[utf8]{inputenc}
\usepackage{libertine}
\usepackage[libertine]{newtxmath}
\usepackage[detect-weight=true]{siunitx}

\title{CHEM 264 Assignment 1}
\author{Brandon Tsang}
\date{May 20, 2020}

\setlength\parindent{0pt}
\definecolor{green}{HTML}{3EAD0A}
\newcommand\osplus{{\ooalign{\hidewidth\(\bigcirc\)\hidewidth\cr\(+\)}}}
\newcommand\osminus{{\ooalign{\hidewidth\(\bigcirc\)\hidewidth\cr\(-\)}}}
\setchemfig{atom sep=1.6em, arrow offset=12pt}

\begin{document}
    \maketitle
    \textbf{\boldmath For each of the following species, draw \underline{all} valid resonance structures and identify the most important resonance contributor. Use curved arrows to show (mental) movement of electron pairs. \begin{center}\chemfig{\charge{90=\:,150:2pt={\tiny\(\osminus\)}}{C}H_2NO_2}\hspace{1in}\chemfig{H-[4]N*6(-=-\charge{180:4pt={\tiny\(\osplus\)}}{}---)}\hspace{1in}\chemfig{[:-54]N*5((-H)-O--(<:NH_2)-(=O)-)}\end{center}}
    \par
    \setchemfig{atom sep=3em}
    Resonance structures for the first one:
    \begin{center}
        \schemestart
            \chemfig{\charge{180=\:,125:4pt={\tiny\(\osminus\)}}{C}(-[2]H)(-[6]H)-\charge{120:4pt={\tiny\(\osplus\)}}{N}(=[:60]\charge{120=\:,0=\:}{O})(-[:-60]\charge{30=\:,-60=\:,-150=\:,45:4pt={\tiny\(\osminus\)}}{O})}
            \arrow{<->}
            \chemfig{\charge{180=\:,125:4pt={\tiny\(\osminus\)}}{C}(-[2]H)(-[6]H)-\charge{120:4pt={\tiny\(\osplus\)}}{N}(-[@{no1}:60]@{o1}\charge{150=\:,60=\:,-30=\:,45:4pt={\tiny\(\osminus\)}}{O})(=[@{no2}:-60]@{o2}\charge{0=\:,-120=\:}{O})}
            \arrow{<->}
            \chemfig{\charge{180=\:,125:4pt={\tiny\(\osminus\)}}{C}(-[2]H)(-[6]H)-\charge{120:4pt={\tiny2 \(\osplus\)}}{N}(-[:60]\charge{150=\:,60=\:,-30=\:,45:4pt={\tiny\(\osminus\)}}{O})(-[@{no3}:-60]@{o3}\charge{30=\:,-60=\:,-150=\:,45:4pt={\tiny\(\osminus\)}}{O})}
            \arrow(--.mid west){<->}
            \chemname{\chemfig{@{c}C(-[2]H)(-[6]H)=[@{cn}]\charge{120:4pt={\tiny\(\osplus\)}}{N}(-[:60]\charge{150=\:,60=\:,-30=\:,45:4pt={\tiny\(\osminus\)}}{O})(-[:-60]\charge{30=\:,-60=\:,-150=\:,45:4pt={\tiny\(\osminus\)}}{O})}}{most important}
        \schemestop
        \chemmove{
            \draw[green, shorten <=2pt, shorten >=5pt] (no1) to[out=150, in=160, min distance=0.6cm] (o1);
            \draw[green, shorten <=4pt, shorten >=4pt] (o2) to[out=200, in=210, min distance=0.6cm] (no2);
            \draw[green, shorten <=4pt, shorten >=4pt] (c) to[out=70, in=90, min distance=0.5cm] (cn);
            \draw[green, shorten <=2pt, shorten >=5pt] (no3) to[out=210, in=200, min distance=0.6cm] (o3);
        }
    \end{center}
    Resonance structures for the second one:
    \begin{center}
        \schemestart
            \chemname{\chemfig{H-[4]\charge{180=\:}{N}*6(-=-\charge{180:4pt={\tiny\(\osplus\)}}{}---)}}{most important}
            \arrow(.mid east--){<->}
            \chemfig{H-[4]\charge{180=\:}{N}*6(-\charge{60:4pt={\tiny\(\osplus\)}}{}-[@{sngl1}]=[@{dbl1}]---)}
            \arrow{<->}
            \chemfig{H-[4]@{n}\charge{60:4pt={\tiny\(\osplus\)}}{N}*6(=[@{dbl2}]-=---)}
        \schemestop
        \chemmove{
            \draw[green, shorten <=2pt, shorten >=4pt] (sngl1) to[out=270, in=-30, min distance=0.5cm] (dbl1);
            \draw[green, shorten <=2pt, shorten >=4pt] (n) to[out=180, in=210, min distance=0.5cm] (dbl2);
        }
    \end{center}
    Resonance structures for the third one:
    \begin{center}
        \schemestart
            \chemname{\chemfig{[:-54]N*5((-H)-O--(<:NH_2)-(=O)-)}}{most important}
            \arrow(.mid east--){<->}
            \chemfig{[:-54]@{n}\charge{70:4pt={\tiny\(\osplus\)}}{N}*5((-H)-O--(<:NH_2)-(-[@{sngl1}]@{o}\charge{45:2pt={\tiny\(\osminus\)}}{O})=[@{dbl}])}
        \schemestop
        \chemmove{
            \draw[green, shorten <=2pt, shorten >=4pt] (n) to[out=-18, in=-54, min distance=0.5cm] (dbl);
            \draw[green, shorten <=2pt, shorten >=2pt] (sngl1) to[out=180, in=180, min distance=0.5cm] (o);
        }
    \end{center}
\end{document}
