\documentclass[11pt]{article}

\usepackage{amsmath}
\usepackage{booktabs}
\usepackage[outline]{contour}
\usepackage{enumitem}
\usepackage[T1]{fontenc}
\usepackage[margin=1in]{geometry}
\usepackage{graphicx}
\usepackage[utf8]{inputenc}
\usepackage{libertine}
\usepackage{mathtools}
\usepackage[libertine]{newtxmath}
\usepackage{pgfplots}
\usepackage{suffix}

\title{PHYS 234 Assignment 5}
\author{Brandon Tsang}
\date{June 19, 2020}

\pgfplotsset{compat=1.16}
% Use \uline{text} to underline text. It ducks the line behind descenders.
\newcommand\uline[1]{\underline{\smash{#1}}\llap{\contour{white}{#1}}}
\DeclareMathOperator{\diff}{d}
% d#1 / d#2. For example, \dd{y}{x} gives dy/dx.
\newcommand\dd[2]{\frac{\diff #1}{\diff #2}}
% d / d#1. For example, \dd*{t} gives d/dt.
\WithSuffix\newcommand\dd*[1]{\frac{\diff{}}{\diff #1}\,}
\DeclarePairedDelimiter\abs{\lvert}{\rvert}

\begin{document}
    \maketitle
    \begin{enumerate}[label=\textbf{\arabic*.}]
        \item{
            \textbf{\boldmath A beam of identical neutral particles with spin \(\frac 1 2\) travels along the \(y\) axis. The beam passes through a series of two Stern-Gerlach spin analyzing magnets, each of which is designed to analyze the spin component along the \(z\) axis. The first Stern-Gerlach analyzer allows only particles with spin up (along the \(z\) axis) to pass through. The second Stern-Gerlach analyzer allows only particles with spin down (along the \(z\) axis) to pass through. The particles travel at a speed \(v\) between the two analyzers, which are separated by a region of length \(d\) in which there is a uniform magnetic field \(B_0\) pointing in the \(x\) direction. Determine the smallest value of \(d\) such that 25\% of the particles transmitted by the first analyzer are transmitted by the second analyzer.}
        }
    \end{enumerate}
\end{document}
