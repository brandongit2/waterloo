\documentclass[11pt]{article}

\usepackage{amsmath}
\usepackage{bm} % For \boldmath command
\usepackage{booktabs}
\usepackage{enumitem}
\usepackage{float}
\usepackage[margin=1in]{geometry}
\usepackage{libertine}
\usepackage[libertine]{newtxmath}
\usepackage{pgfplots}
\usepackage{siunitx}

\pgfplotsset{compat=1.16}
\usetikzlibrary{arrows.meta}
\usetikzlibrary{bending} % For 'flex' option on arrow tips
\usetikzlibrary{calc}
\usetikzlibrary{decorations}
\usetikzlibrary{positioning}
\pgfarrowsdeclarecombine{dist}{dist}{stealth}{stealth}{|}{|}

\sisetup{separate-uncertainty=true}

\title{Experiment 7: Standing Waves on a Wire}
\author{
    Brandon Tsang \\
    PHYS 122L-002
}
\date{January 29, 2020}

\begin{document}
\maketitle
\section*{Goals}
    \textit{The following are quoted directly from PHYS 122L Lab Manual (Department of Physics and Astronomy, 2020).}
    \begin{itemize}
        \item To produce and observe standing waves on a wire.
        \item To demonstrate that the fundamental frequencies are proportional to the square root of the tension of the wire when the length is fixed.
        \item To observe the harmonic frequencies of a wire of fixed length and fixed tension.
        \item To demonstrate that the fundamental frequencies of standing waves on a wire depend inversely on the length when the tension is fixed.
        \item To analyze data graphically.
    \end{itemize}
\section*{\boldmath Part A: Investigation of the dependence of $f_1$ on $T$ with $L$ constant}
    \subsection*{Experiment Summary}
        
    \subsection*{Results}
        The length $L$ of the wire: \SI{1.0}{\meter}
        \begin{table}[H]
            \centering
            \caption{Data for fundamental frequencies for different masses}
            \begin{tabular}{c c c c c c c}
                \toprule
                $M$ (\si{\kilogram}) & $f_1$ (\si{\hertz}) & $\Delta f_1$ (\si{\hertz}) & $f_1^2$ (\si{\hertz\squared}) & $\Delta f_1^2$ (\si{\hertz\squared}) & $\log(M)$ & $\log(f_1)$ \\
                \midrule
                0.2 & 14.3 & 0.1 & 204 & 2.9 & -0.699 & 1.16 \\
                0.4 & 20.1 & 0.1 & 404 & 4.0 & -0.398 & 1.30 \\
                0.6 & 24.6 & 0.1 & 605 & 4.9 & -0.222 & 1.39 \\
                0.8 & 28.3 & 0.1 & 801 & 5.7 & -0.0969 & 1.45 \\
                1.0 & 31.7 & 0.1 & 1010 & 6.3 & 0.000 & 1.50 \\
                1.2 & 34.7 & 0.1 & 1200 & 6.9 & 0.0792 & 1.54 \\
                1.5 & 38.8 & 0.1 & 1510 & 7.8 & 0.176 & 1.59 \\
                \bottomrule
            \end{tabular}
        \end{table}
        Below is the plot of $f_1^2$ vs. $M$. The error bars are too small to be shown.
        \begin{figure}[H]
            \centering
            \caption{Relationship between fundamental frequency and mass of weight}
            \begin{tikzpicture}
                \begin{axis}[
                    xlabel={$M$ (\si{\kilogram})},
                    ylabel={$f_1^2$ (\si{\hertz\squared})}
                ]
                    \addplot[
                        black, only marks, mark=square*,
                        error bars/.cd,
                        y dir=both, y explicit
                    ] coordinates {
                        (1.5,1505.4) +- (7.8)
                        (1.2,1204.1) +- (0,6.9)
                        (1,1004.9) +- (0,6.3)
                        (0.8,800.9) +- (0,5.7)
                        (0.6,605.2) +- (0,4.9)
                        (0.4,404) +- (0,4)
                        (0.2,204.5) +- (0,2.9)
                    };
                    \addplot[dashed, domain=0:1.6, samples=5] {1000*x+3.54};
                \end{axis}
            \end{tikzpicture}
        \end{figure}
        Slope of the above graph: \SI{1000}{\hertz\squared\per\kilogram}
        \par\noindent
        The slope $m$ should be equal to $\frac{g}{4L^2\mu}$. Solving for $\mu$:
        \begin{align*}
            m&=\frac{g}{4L^2\mu} \\
            4L^2\mu&=\frac{g}{m} \\
            \mu&=\frac{g}{4L^2m} \\
            &=\frac{\SI{9.8}{\meter\per\square\second}}{4(\SI{1.0}{\meter})^2(\SI{1000}{\hertz\squared\per\kilogram})} \\
            &=\SI{0.00245}{\kilogram\per\meter}
        \end{align*}
        Logarithmic plot of $f_1$ vs $M$:
        \begin{figure}[H]
            \centering
            \caption{Logarithmic relationship between fundamental frequency and mass of weight}
            \begin{tikzpicture}
                \begin{loglogaxis}[
                    xlabel={$M$ (\si{\kilogram})},
                    ylabel={$f_1$ (\si{\hertz})}
                ]
                    \addplot[black, only marks, mark=square*] coordinates {
                        (14.3,0.2)
                        (20.1,0.4)
                        (24.6,0.6)
                        (28.3,0.8)
                        (31.7,1)
                        (34.7,1.2)
                        (38.8,1.5)
                    };
                \end{loglogaxis}
            \end{tikzpicture}
        \end{figure}
        $M=1$ intercept: \SI{31.6}{\hertz}
        \par\noindent
        This intercept (call it $b$) should be equal to $\frac{1}{2L}\sqrt{\frac{g}{\mu}}$. Solving for $\mu$:
        \begin{align*}
            b&=\frac{1}{2L}\sqrt{\frac{g}{\mu}} \\
            2Lb&=\sqrt{\frac{g}{\mu}} \\
            4L^2b^2&=\frac{g}{\mu} \\
            \mu&=\frac{g}{4L^2b^2} \\
            &=\frac{\SI{9.8}{\meter\per\square\second}}{4(\SI{1.0}{\meter})^2(\SI{31.6}{\hertz})^2} \\
            &=\SI{0.00245}{\kilogram\per\meter}
        \end{align*}
\section*{Parth B: Investigation of harmonic frequencies}
    \subsection*{Results}
        The following results are using a \SI{0.5}{\kilogram} weight:
        \begin{center}
            \begin{tabular}{c c c c}
                \toprule
                $n$ & $f_n$ & $\Delta f_n$ & $\frac{f_n}{f_1}$ \\
                \midrule
                1 & 22.7 & 0.1 & 1.00 \\
                2 & 44.4 & 0.1 & 1.96 \\
                3 & 67.7 & 0.1 & 2.98 \\
                4 & 90.8 & 0.1 & 4.00 \\
                5 & 113 & 1 & 4.98 \\
                6 & 136 & 1 & 5.99 \\
                7 & 160 & 1 & 7.05 \\
                \bottomrule
            \end{tabular}
        \end{center}
\section*{\boldmath Part C: Investigation of the dependence of $f_1$ on $L$ with $T$ fixed}
    \subsection*{Results}
        \begin{table}[H]
            \centering
            \caption{Using a \SI{500}{\gram} mass, the fundamental frequencies at different distances}
            \begin{tabular}{c c c c c c c c}
                \toprule
                $L$ (\si{\meter}) & $\Delta L$ (\si{\meter}) & $L^{-1}$ (\si{\per\meter}) & $\Delta L^{-1}$ (\si{\per\meter}) & $f_1$ (\si{\hertz}) & $\Delta f_1$ (\si{\hertz}) & $\log(L)$ & $\log(f_1)$ \\
                \midrule
                0.2 & 0.1 & 5.00 & 2.50 & 113 & 1 & -0.70 & 2.05 \\
                0.3 & 0.1 & 3.33 & 1.11 & 74.6 & 0.1 & -0.52 & 1.87 \\
                0.4 & 0.1 & 2.50 & 0.625 & 57.5 & 0.1 & -0.40 & 1.76 \\
                0.5 & 0.1 & 2.00 & 0.400 & 44.8 & 0.1 & -0.30 & 1.65 \\
                0.6 & 0.1 & 1.67 & 0.278 & 38.9 & 0.1 & -0.22 & 1.59 \\
                0.7 & 0.1 & 1.43 & 0.204 & 32.2 & 0.1 & -0.15 & 1.51 \\
                0.8 & 0.1 & 1.25 & 0.156 & 28.2 & 0.1 & -0.10 & 1.45 \\
                0.9 & 0.1 & 1.11 & 0.123 & 25.1 & 0.1 & -0.05 & 1.40 \\
                1.0 & 0.1 & 1.00 & 0.100 & 22.7 & 0.1 & 0.00 & 1.36 \\
                \bottomrule
            \end{tabular}
        \end{table}
\end{document}
