\documentclass[11pt]{article}

\usepackage{amsmath}
\usepackage{booktabs}
\usepackage{enumitem}
\usepackage[T1]{fontenc}
\usepackage[margin=1in]{geometry}
\usepackage{graphicx}
\usepackage[utf8]{inputenc}
\usepackage{libertine}
\usepackage[libertine]{newtxmath}
\usepackage[detect-weight=true]{siunitx}

\title{PHYS 124 Final Exam Question 4}
\author{Brandon Tsang}
\date{April 23, 2020}

\begin{document}
    \maketitle
    \begin{enumerate}[label=\textbf{\arabic*.}, start=4]
        \item{
            \textbf{\boldmath Certain structures called quantum wires behave as metals in which its gas of electrons is confined to a line (one spatial dimension). The density of states for a quantum wire of length \(L\) is \(g(E)=\frac{l}{\hbar\pi}\sqrt{\frac{m}{2E}}\) where \(m\) is the mass of an electron.}
            \begin{enumerate}[label=\textbf{(\alph*)}]
                \item{
                    \textbf{\boldmath Find the number of electrons \(N\) in a quantum wire in terms of \(L\), \(E\), and \(m\).}
                    \par
                    Using the Fermi-Dirac distribution and the density of states, we can calculate the number of electrons. We can define a variable \(dN\) which represents the number of electrons in an infinitessimal energy band \(dE\):
                    \begin{align*}
                        dN&=\frac{g(E)}{\exp\left(\frac{E-E_\mathrm{F}}{kT}\right)+1}\,dE \\
                        &=\frac{\frac{l}{\hbar\pi}\sqrt{\frac{m}{2E}}}{\exp\left(\frac{E-E_\mathrm{F}}{kT}\right)+1}\,dE
                    \end{align*}
                    where \(E_\mathrm{F}\) is the Fermi energy. When \(T\) approches zero, this becomes \[\frac{l}{\hbar\pi}\sqrt{\frac{m}{2E}}.\]
                    I can integrate this from 0 to \(E_\mathrm{F}\):
                    \begin{align*}
                        N&=\int_0^{E_\mathrm{F}}\frac{l}{\hbar\pi}\sqrt{\frac{m}{2E}}\,dE \\
                        &=\frac{\sqrt m}{\sqrt{2}\hbar\pi}\int_0^{E_\mathrm{F}}\frac{1}{\sqrt{E}}\,dE \\
                        &=\frac{\sqrt m}{\sqrt{2}\hbar\pi}\left[2E^{1/2}\right]_0^{E_\mathrm{F}} \\
                        &=\frac{2\sqrt m}{\sqrt{2}\hbar\pi}E_\mathrm{F}^{1/2}
                    \end{align*}
                }
            \end{enumerate}
        }
    \end{enumerate}
\end{document}
