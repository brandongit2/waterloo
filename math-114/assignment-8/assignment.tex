\documentclass[11pt]{article}

\usepackage{amsmath}
\usepackage{bm} % For \boldmath command
\usepackage{booktabs}
\usepackage{enumitem}
\usepackage[T1]{fontenc}
\usepackage[margin=1in]{geometry}
\usepackage[utf8]{inputenc}
\usepackage{libertine}
\usepackage[libertine]{newtxmath}

\title{MATH 114 Assignment 8}
\author{Brandon Tsang}
\date{March 26, 2020}

\begin{document}
\maketitle
\begin{enumerate}[label=\textbf{\arabic*.}]
    \item{
          \begin{enumerate}[label=\textbf{(\alph*)}]
              \item{
                    \textbf{\boldmath A matrix $M$ transformed the figure $A$ to the figure $B$. Calculate $\det(M)$.}
                    \par
                    The area of A (found by counting grid squares) is 5, and the area of B is 2. Therefore, the transformation from $M$ decreased the area by a factor of $\frac{2}{5}$, and $\det(M)=\frac{2}{5}$.
                    }
              \item{
                    \textbf{\boldmath Given $N=\begin{bmatrix}7 & 3 \\ 8 & 4\end{bmatrix}$, calculate $\det(N)$.}
                    \par
                    $\det(N)$ is calculated as follows:
                    \begin{align*}
                        \det(N)=\begin{vmatrix}7 & 3 \\ 8 & 4\end{vmatrix} & =(7)(4)-(3)(8) \\
                                                          & =28-24         \\
                                                          & =4
                    \end{align*}
                    }
              \item{
                    \textbf{
                    \boldmath In the previous assignment, we created the following matrix $P$:
                    {\small
                    $$
                        P=
                        \begin{bmatrix}
                            1 & 0 & 0 & 4 \\
                            0 & 1 & 0 & 0 \\
                            0 & 0 & 1 & 0 \\
                            0 & 0 & 0 & 1
                        \end{bmatrix}
                        \begin{bmatrix}
                            \frac{3}{\sqrt{10}}  & \frac{1}{\sqrt{10}} & 0 & 0 \\
                            -\frac{1}{\sqrt{10}} & \frac{3}{\sqrt{10}} & 0 & 0 \\
                            0                    & 0                   & 1 & 0 \\
                            0                    & 0                   & 0 & 1
                        \end{bmatrix}
                        \begin{bmatrix}
                            \cos\theta  & 0 & \sin\theta & 0 \\
                            0           & 1 & 0          & 0 \\
                            -\sin\theta & 0 & \cos\theta & 0 \\
                            0           & 0 & 0          & 1
                        \end{bmatrix}
                        \begin{bmatrix}
                            \frac{3}{\sqrt{10}} & -\frac{1}{\sqrt{10}} & 0 & 0 \\
                            \frac{1}{\sqrt{10}} & \frac{3}{\sqrt{10}}  & 0 & 0 \\
                            0                   & 0                    & 1 & 0 \\
                            0                   & 0                    & 0 & 1
                        \end{bmatrix}
                        \begin{bmatrix}
                            1 & 0 & 0 & -4 \\
                            0 & 1 & 0 & 0  \\
                            0 & 0 & 1 & 0  \\
                            0 & 0 & 0 & 1
                        \end{bmatrix}
                    $$
                    }
                    Calculate $\det(P)$.
                    }
                    \par
                    First, we find the determinants of the matrices which make up $P$, which we will call $P_1$, $P_2$, $P_3$, $P_4$, and $P_5$. We will utilize the shortcut method discussed in class.
                    \begin{align*}
                        \det(P_1)=
                        \begin{vmatrix}
                            1 & 0 & 0 & 4 \\
                            0 & 1 & 0 & 0 \\
                            0 & 0 & 1 & 0 \\
                            0 & 0 & 0 & 1
                        \end{vmatrix} & =(1)(1)(1)(1) \\
                                                  & =1
                    \end{align*}
                    \begin{align*}
                        \det(P_2)=
                        \begin{vmatrix}
                            \frac{3}{\sqrt{10}}  & \frac{1}{\sqrt{10}} & 0 & 0 \\
                            -\frac{1}{\sqrt{10}} & \frac{3}{\sqrt{10}} & 0 & 0 \\
                            0                    & 0                   & 1 & 0 \\
                            0                    & 0                   & 0 & 1
                        \end{vmatrix} & \sim
                        \begin{vmatrix}
                            \frac{3}{\sqrt{10}} & \frac{1}{\sqrt{10}}                      & 0 & 0 \\
                            0                   & \frac{3}{\sqrt{10}}+\frac{1}{3\sqrt{10}} & 0 & 0 \\
                            0                   & 0                                        & 1 & 0 \\
                            0                   & 0                                        & 0 & 1
                        \end{vmatrix}                                                                      \\
                                                  & \sim
                        \begin{vmatrix}
                            \frac{3}{\sqrt{10}} & \frac{1}{\sqrt{10}}   & 0 & 0 \\
                            0                   & \frac{10}{3\sqrt{10}} & 0 & 0 \\
                            0                   & 0                     & 1 & 0 \\
                            0                   & 0                     & 0 & 1
                        \end{vmatrix}                                                                      \\
                                                  & =\left(\frac{3}{\sqrt{10}}\right)\left(\frac{10}{3\sqrt{10}}\right) \\
                                                  & =\frac{30}{30}                                                      \\
                                                  & =1
                    \end{align*}
                    \begin{align*}
                        \begin{split}
                            \det(P_3)=
                            \begin{vmatrix}
                                \cos\theta  & 0 & \sin\theta & 0 \\
                                0           & 1 & 0          & 0 \\
                                -\sin\theta & 0 & \cos\theta & 0 \\
                                0           & 0 & 0          & 1
                            \end{vmatrix}
                            & =
                            \cos\theta\begin{vmatrix}
                                1 & 0          & 0 \\
                                0 & \cos\theta & 0 \\
                                0 & 0          & 1
                            \end{vmatrix}
                            -
                            0\begin{vmatrix}
                                0           & 0          & 0 \\
                                -\sin\theta & \cos\theta & 0 \\
                                0           & 0          & 1
                            \end{vmatrix} \\
                            &\qquad+
                            \sin\theta\begin{vmatrix}
                                0           & 1 & 0 \\
                                -\sin\theta & 0 & 0 \\
                                0           & 0 & 1
                            \end{vmatrix}
                            -
                            0\begin{vmatrix}
                                0           & 1 & 0          \\
                                -\sin\theta & 0 & \cos\theta \\
                                0           & 0 & 0
                            \end{vmatrix}
                        \end{split}                     \\
                         & =\cos\theta(1)(\cos\theta)(1)-0-
                        \sin\theta\begin{vmatrix}
                            -\sin\theta & 0 & 0 \\
                            0           & 1 & 0 \\
                            0           & 0 & 1
                        \end{vmatrix}
                        -0                                             \\
                         & =\cos^2\theta-\sin\theta(-\sin\theta)(1)(1) \\
                         & =\cos^2\theta+\sin^2\theta                  \\
                         & = 1
                    \end{align*}
                    \begin{align*}
                        \det(P_1)=
                        \begin{vmatrix}
                            1 & 0 & 0 & -4 \\
                            0 & 1 & 0 & 0  \\
                            0 & 0 & 1 & 0  \\
                            0 & 0 & 0 & 1
                        \end{vmatrix} & =(1)(1)(1)(1) \\
                                                   & =1
                    \end{align*}
                    Finally,
                    \begin{align*}
                        \det(P) & =\det(P_1)\det(P_2)\det(P_3)\det(P_4)\det(P_5) \\
                                & =1\cdot1\cdot1\cdot1\cdot1                     \\
                                & =1
                    \end{align*}
                    }
          \end{enumerate}
          }
          \pagebreak
    \item{
          \begin{enumerate}[label=\textbf{(\alph*)}]
              \item{
                    \textbf{\boldmath Compute the determinant of $Q=\begin{bmatrix}2 & 0 & 0 & 1 \\ -4 & 1 & 0 & -2 \\ -3 & 0 & 1 & 0 \\ -3 & 1 & 0 & -2\end{bmatrix}$. What can you say about $\det(Q^{-1})$?}
                    \par
                    The determinant of $Q$ is calculated as follows:
                    \begin{align*}
                        \det(Q)=
                        \begin{vmatrix}
                            2  & 0 & 0 & 1  \\
                            -4 & 1 & 0 & -2 \\
                            -3 & 0 & 1 & 0  \\
                            -3 & 1 & 0 & -2
                        \end{vmatrix}
                         & \sim
                        \begin{vmatrix}
                            2  & 0 & 0 & 1  \\
                            0  & 1 & 0 & 0  \\
                            0  & 1 & 1 & -2 \\
                            -3 & 1 & 0 & -2
                        \end{vmatrix}             \\
                         & \sim
                        \begin{vmatrix}
                            2 & 0 & 0 & 1            \\
                            0 & 1 & 0 & 0            \\
                            0 & 1 & 1 & -2           \\
                            0 & 1 & 0 & -\frac{1}{2}
                        \end{vmatrix}             \\
                         & \sim
                        \begin{vmatrix}
                            2 & 0 & 0 & 1            \\
                            0 & 1 & 0 & 0            \\
                            0 & 0 & 1 & -2           \\
                            0 & 0 & 0 & -\frac{1}{2}
                        \end{vmatrix}             \\
                         & =(2)(1)(1)\left(-\frac{1}{2}\right) \\
                         & =-1
                    \end{align*}
                    }
              \item{
                    \textbf{\boldmath Compute the determinant of $R=\begin{bmatrix}1 & 0 & 3 & 3 \\ 0 & 1 & 1 & 1 \\ 0 & 0 & 1 & 1 \\ 2 & 0 & 6 & 6\end{bmatrix}$. What can you say about $\det(R^{-1})$?}
                    \par
                    The determinant of $R$ is calculated as follows:
                    \begin{align*}
                        \det(R)=
                        \begin{vmatrix}
                            1 & 0 & 3 & 3 \\
                            0 & 1 & 1 & 1 \\
                            0 & 0 & 1 & 1 \\
                            2 & 0 & 6 & 6
                        \end{vmatrix}
                         & \sim
                        \begin{vmatrix}
                            1 & 0 & 3 & 3 \\
                            0 & 1 & 1 & 1 \\
                            0 & 0 & 1 & 1 \\
                            0 & 0 & 0 & 0
                        \end{vmatrix}
                    \end{align*}
                    Since the bottom row is all zeroes, this matrix has no determinant.
                    }
          \end{enumerate}
          }
\end{enumerate}
\end{document}
