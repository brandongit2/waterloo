\documentclass[11pt]{article}

\usepackage{amsmath}
\usepackage{braket}
\usepackage{booktabs}
\usepackage[outline]{contour}
\usepackage{enumitem}
\usepackage[T1]{fontenc}
\usepackage[margin=1in]{geometry}
\usepackage{graphicx}
\usepackage[utf8]{inputenc}
\usepackage{libertine}
\usepackage{mathtools}
\usepackage[libertine]{newtxmath}
\usepackage{parskip}
\usepackage[detect-weight=true]{siunitx}

\title{PHYS 234 Assignment 1}
\author{Brandon Tsang}
\date{May 18, 2020}

\contourlength{0.1em}

\DeclareMathOperator\conj{conj}
\newcommand\abs[1]{\lvert#1\rvert}
\newcommand\Abs[1]{\left\lvert#1\right\rvert}
\newcommand\uline[1]{\underline{\smash{#1}}\llap{\contour{white}{#1}}}

\newenvironment{amatrix}[1]{%
    \left[\begin{array}{@{}*{#1}{c}|c@{}}
}{%
    \end{array}\right]
}

\begin{document}
    \maketitle
    \begin{enumerate}[label=\textbf{\arabic*.}]
        \item{
            \textbf{\uline{Relationship between trigonometric functions and complex exponentials}}
            \begin{enumerate}[label=\textbf{(\alph*)}]
                \item{
                    \textbf{\boldmath Starting from the power series representation of the exponential function \(e^x\), derive Euler's formula: \[e^{i\theta}=\cos\theta+i\sin\theta\] where \(i=\sqrt{-1}\). \[e^x=\sum_{n=0}^\infty\frac{x^n}{n!},\quad\cos x=\sum_{n=0}^\infty\frac{(-1)^nx^{2n}}{(2n)!},\quad\sin x=\sum_{n=0}^\infty\frac{(-1)^nx^{2n+1}}{(2n+1)!}\]}
                    \par
                    To start, it could be useful to write out the infinite sums in full:
                    \begin{align*}
                        e^x=\sum_{n=0}^\infty\frac{x^n}{n!}&=\frac{x^0}{0!}+\frac{x^1}{1!}+\frac{x^2}{2!}+\frac{x^3}{3!}+\frac{x^4}{4!}+\frac{x^5}{5!}+\frac{x^6}{6!}+\frac{x^7}{7!}+\frac{x^8}{8!}+\frac{x^9}{9!}+\ldots \\
                        &=1+x+\frac{x^2}{2}+\frac{x^3}{3!}+\frac{x^4}{4!}+\frac{x^5}{5!}+\frac{x^6}{6!}+\frac{x^7}{7!}+\frac{x^8}{8!}+\frac{x^9}{9!}+\ldots
                    \end{align*}
                    \begin{align*}
                        \cos x=\sum_{n=0}^\infty\frac{(-1)^nx^{2n}}{(2n)!}&=\frac{(-1)^0x^0}{0!}+\frac{(-1)^1x^2}{2!}+\frac{(-1)^2x^4}{4!}+\frac{(-1)^3x^6}{6!}+\frac{(-1)^4x^8}{8!}+\ldots \\
                        &=1-\frac{x^2}{2}+\frac{x^4}{4!}-\frac{x^6}{6!}+\frac{x^8}{8!}-\ldots
                    \end{align*}
                    \begin{align*}
                        \sin x=\sum_{n=0}^\infty\frac{(-1)^nx^{2n+1}}{(2n+1)!}&=\frac{(-1)^0x^1}{1!}+\frac{(-1)^1x^3}{3!}+\frac{(-1)^2x^5}{5!}+\frac{(-1)^3x^7}{7!}+\frac{(-1)^4x^9}{9!}+\ldots \\
                        &=x-\frac{x^3}{3!}+\frac{x^5}{5!}-\frac{x^7}{7!}+\frac{x^9}{9!}-\ldots
                    \end{align*}
                    When written out like this, it is easy to see that the expansion of \(\cos x\) is extremely similar to that of \(\sin x\), but \(\cos x\) contains all the even terms, while \(\sin x\) contains all the odd terms. Another observation to make is that the expansion of \(e^x\) is astoundingly similar to \(\cos x+\sin x\):
                    \[\cos x+\sin x=1+x-\frac{x^2}{2}-\frac{x^3}{3!}+\frac{x^4}{4!}+\frac{x^5}{5!}-\frac{x^6}{6!}-\frac{x^7}{7!}+\frac{x^8}{8!}+\frac{x^9}{9!}-\ldots\]
                    The only difference is the signs on some of the terms. The pattern appears to be \(++--++--++--\ldots\), which is suspiciously similar to the signs of the powers of \(i\):
                    \[i^0=1,\ i^1=i,\ i^2=-1,\ i^3=-i,\ i^4=1,\ i^5=i,\ i^6=-1,\ \ldots\]
                    The thought comes to mind: what if instead of \(e^x\), we wrote out \(e^{ix}\) instead? Then, the \(x^n\) portion of that infinite sum would lead to the same sign pattern as in \(\cos x+\sin x\).
                    \begin{align*}
                        e^{ix}&=1+ix+\frac{(ix)^2}{2}+\frac{(ix)^3}{3!}+\frac{(ix)^4}{4!}+\frac{(ix)^5}{5!}+\frac{(ix)^6}{6!}+\frac{(ix)^7}{7!}+\frac{(ix)^8}{8!}+\frac{(ix)^9}{9!}-\ldots \\
                        &=1+ix-\frac{x^2}{2}-\frac{ix^3}{3!}+\frac{x^4}{4!}+\frac{ix^5}{5!}-\frac{x^6}{6!}-\frac{ix^7}{7!}+\frac{x^8}{8!}+\frac{ix^9}{9!}-\ldots
                    \end{align*}
                    Now the only difference between \(e^{ix}\) and \(\cos x+\sin x\) is a factor of \(i\) on the odd terms. But, as we saw earlier, the odd terms come from \(\sin x\)! If we instead write the expansion of \(\cos x+i\sin x\) (multiplying \(\sin x\) by \(i\)), we get the two expressions to exactly match. This leads us to writing the formula
                    \[e^{ix}=\cos x+i\sin x.\]
                }
                \item{
                    \textbf{\boldmath Use Euler's Formula to express the trigonometric functions \(\cos\theta\) and \(\sin\theta\) in terms of the complex exponential functions \(e^{\pm i\theta}\).}
                    \par
                    What happens to Euler's formula when we use \(e^{-ix}\) instead of \(e^{ix}\)?
                    \begin{align*}
                        e^{-ix}&=\cos(-x)+i\sin(-x) \\
                        &=\cos x-i\sin x
                    \end{align*}
                    Notice that the sign of \(\sin x\) changes but the sign of \(\cos x\) does not. Naturally, this means that \[e^{ix}+e^{-ix}=2\cos x,\] and \[e^{ix}-e^{-ix}=2i\sin x.\] Solving each of these for \(\sin x\) and \(\cos x\), we get \[\cos x=\frac{e^{ix}+e^{-ix}}{2}\] and \[\sin x=\frac{e^{ix}-e^{-ix}}{2i}.\]
                }
            \end{enumerate}
        }
    \end{enumerate}
\end{document}
