\documentclass[11pt]{article}

\usepackage{amsmath}
\usepackage{booktabs}
\usepackage{enumitem}
\usepackage[T1]{fontenc}
\usepackage[margin=1in]{geometry}
\usepackage{graphicx}
\usepackage[utf8]{inputenc}
\usepackage{libertine}
\usepackage[libertine]{newtxmath}
\usepackage{siunitx}

\title{PHYS 124 Final Exam Question 2}
\author{Brandon Tsang}
\date{April 23, 2020}

\begin{document}
    \maketitle
    \begin{enumerate}[label=\textbf{\arabic*.}, start=2]
        \item{
            \begin{enumerate}[label=\textbf{(\alph*)}]
                \item{
                    \textbf{\boldmath The Friedmann equations describing our universe at large scales are
                    \begin{equation}
                        \label{eqn:1}
                        \frac{1}{2}\left(\frac{da(t)}{dt}\right)^2-\frac{4\pi G\rho(t)}{3}a^2(t)=-\frac{kc^2}{2l^2}
                    \end{equation}
                    and
                    \begin{equation}
                        \label{eqn:2}
                        \frac{d}{dt}(\rho(t)a^3(t))=-p(t)\frac{d}{dt}a^3(t)
                    \end{equation}
                    for the scale factor \(a(t)\). Explain the meaning of the quantities \(k\), \(l\), \(\rho\), and \(p\) in these equations.}
                    \par
                    \(k\) can be either \(+1\), \(0\), or \(-1\), and it defines the shape of the universe.
                    \par
                    \(\rho\) is the density of energy in the universe.
                    \par
                    \(p\) is the pressure of energy in the universe.
                }
                \item{
                    \textbf{\boldmath Suppose that the scale factor \(a(t)=a_0t^{1/2}\). If \(k=0\), find the dependence of \(\rho\) and \(p\) on \(t\). What cosmic era does this correspond to?}
                    \par
                    Using equation \ref{eqn:1}:
                    \begin{align*}
                        \frac{1}{2}\left(\frac{da(t)}{dt}\right)^2-\frac{4\pi G\rho(t)}{3}a^2(t)&=-\frac{kc^2}{2l^2} \\
                        \frac{1}{2}\left(\frac{1}{2}a_0t^{-1/2}\right)^2-\frac{4\pi G\rho(t)}{3}(a_0^2t^{1/2})&=0 \\
                        \frac{1}{8t^{3/2}}a_0^2-\frac{4\pi G\rho(t)}{3}(a_0^2t^{1/2})&=0 \\
                        \frac{1}{8t^{3/2}}&=\frac{4\pi G\rho(t)}{3}t^{1/2} \\
                        \frac{3}{8}&=4\pi G\rho(t)t^2 \\
                        \rho(t)&=\frac{3}{32\pi Gt^2}
                    \end{align*}
                    Then, with equation \ref{eqn:2}:
                    \begin{align*}
                        \frac{d}{dt}(\rho(t)a^3(t))&=-p(t)\frac{d}{dt}a^3(t) \\
                        \frac{d}{dt}\left(\frac{3}{32\pi Gt^2}(a_0^3t^{3/2})\right)&=-p(t)\frac{d}{dt}(a_0^3t^{3/2}) \\
                        \frac{3a_0^3}{32\pi G}\frac{d}{dt}\left(t^{-1/2}\right)&=-p(t)\left(\frac{3}{2}a_0^3t^{1/2}\right) \\
                        \frac{1}{32\pi G}t^{-3/2}&=p(t)t^{1/2} \\
                        p(t)&=\frac{1}{32\pi Gt^2}
                    \end{align*}
                }
            \end{enumerate}
            \textit{[The rest of this question was skipped.]}
        }
    \end{enumerate}
\end{document}
